\chapter{Aufgabenstellung}
Die Aufgabe dieser Arbeit lässt sich in drei Teile aufteilen:
\begin{enumerate}
	\item Erstellung eines Goldstandards zur Quantifizierung der Qualität der Klassifizierung von Menüseiten
	\item Bestimmung eines Klassifikationsalgorithmus, welcher anhand der Daten des Gold Standards ein möglichst gutes Ergebnis erzielt
	\item Erstellung eines Webcrawlers und einer Webapplikation
\end{enumerate}
Bei der Reihenfolge ist zu beachten, dass die Erstellung des Webcrawlers bereits zu Beginn erfolgt.
Dies aus dem Grund, dass die damit gecrawlten Daten als Grundlage zur Erstellung des Gold Standards dienen.
\section{Scope}
\label{chap:scope}
Um diese Aufgaben zu erfüllen, werden verschiedene Einschränkungen vorgenommen:
\begin{itemize}
	\item Als Seed für den Webcrawler werden nur Websites von Restaurants verwendet
	\item Es werden nur Webpages in HTML analysiert, keine PDF-Dateien
	\item Nur Webpages in deutscher Sprache (mit Ausnahme von fremdsprachigen Speisebezeichnungen) werden berücksichtigt
	\item Die Klassifikation findet anhand des Textes einer Webpage statt, die HTML-Struktur wird nicht berücksichtigt
\end{itemize}
Durch diese Einschränkungen wird die Breite der Arbeit verkleinert, sodass diese im vorgegebenen Zeitrahmen zu bewältigen ist.