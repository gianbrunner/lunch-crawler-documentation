%Dokumentklasse
\documentclass[a4paper,12pt]{scrreprt}
\usepackage[left= 2.5cm,right = 2cm, bottom = 4 cm]{geometry}
%\usepackage[onehalfspacing]{setspace}
% ============= Packages =============


% Dokumentinformationen
\usepackage[
	pdftitle={Bachelorarbeit},
	pdfsubject={},
	pdfauthor={Santoro Sandro, Brunner Gian},
	pdfkeywords={},	
	%Links nicht einrahmen
	hidelinks]{hyperref}

% Standard Packages
\usepackage[utf8]{inputenc}
\usepackage[ngerman]{babel}
\usepackage[T1]{fontenc}
\usepackage{graphicx}
\graphicspath{{img/}}
\usepackage{lmodern}
\usepackage{color}
\usepackage{siunitx}
\usepackage{tikz}
\usepackage{caption}
\usepackage{placeins}
\usepackage{tabularx}
\usepackage{arydshln}
\usepackage{pdfpages}
\usepackage{listings}
\usepackage{url}
\usepackage{mathtools}
\usepackage{makecell}
\usepackage{multirow}
\usepackage{booktabs}
\usepackage{float}
\sisetup{detect-weight=true, detect-family=true}


% ============= Kopf- und Fußzeile =============

\usepackage[automark,headsepline]{scrlayer-scrpage} %footsepline
\pagestyle{scrheadings}
\automark[chapter]{chapter}
\clearscrheadfoot

\ihead{\headmark}				%Kopfzeile innen mit Kapitel
\chead{}						%Kopfzeile Mitte
\ohead[\pagemark]{\pagemark}	%Kopfzeile aussen mit Seitennummer

%\ifoot{}						%Fußzeile innen
%\cfoot{}						%Fußzeile Mitte
%\ofoot{}						%Fußzeile außen

\renewcommand*\chapterpagestyle{scrheadings}

%============== Abstände von Kopfzeile zu Text überall gleich ============%
\renewcommand*\chapterheadstartvskip{\vspace*{-\topskip}}
\renewcommand*\chapterheadendvskip{%
\vspace*{1\baselineskip plus .1\baselineskip minus .167\baselineskip}}


% zusätzliche Schriftzeichen der American Mathematical Society
\usepackage{amsfonts}
\usepackage{amsmath}
\usepackage{cleveref}

%nicht einrücken nach Absatz
\setlength{\parindent}{0pt}

%============== Package für Dateistruktur-Baum ============%
\usepackage{forest}

% ============= Package Einstellungen & Sonstiges ============= 
%Besondere Trennungen
\hyphenation{De-zi-mal-tren-nung}


%=========== Abbildungen fangen mit 1 an ===========%
\renewcommand*{\thefigure}{\arabic{figure}}


% ============= Dokumentbeginn =============
\begin{document}
\setcounter{secnumdepth}{4}	%Bis wo das nummeriert wird (4te Ebene)
\setcounter{tocdepth}{4}

%Seite ohne Kopf- und Fußzeile sowie Seitenzahl
\thispagestyle{empty}

\renewcommand*\chapterpagestyle{scrheadings}
\includegraphics[scale=0.5]{img/ntb.jpg}

\begin{center}
\vspace*{4 cm}
\textbf{\Huge Bachelorarbeit}
\end{center}

\begin{center}
\vspace*{1.5 cm}
\LARGE{\textsf{Menü-Crawler\\}}
\large{\textsf{Prototyp einer Search Engine zur Suche von Speisen und Menüs in der deutschsprachigen Schweiz\\}}
\end{center}


\begin{center}
\vspace*{2 cm}
\textbf{ Studierende:} {\textbf{Sandro Santoro \& Gian Brunner\\}}
\end{center}

\begin{center}
\vspace*{1 cm}
Abgabedatum: 9. August 2019
\vspace*{1 cm}
\end{center}


\begin{center}
\begin{tabular}{lll}
\textbf{Referent:} & & Prof. Corsin Capol\\
\medskip
\textbf{Korreferent:} & & Lukas Toggenburger\\
\end{tabular}
\end{center}


\pagestyle{scrheadings}
\pagenumbering{Roman}

\addchap{Abstract}
\addsec{Abstract Deutsch}

Ziel dieser Arbeit ist das Erstellen einer Suchmaschine, über welche sich Menüs und Speisen suchen lassen.
Die Grundlage einer solchen Suchmaschine sind Websites von Restaurants, welche relevante Speiseinformationen beinhalten.
Im Kontext dieser Bachelorarbeit wurde manuell ein Gold-Standard aus Restaurantseiten zusammengestellt.
Für die Erstellung des Gold-Standards wurde eigens ein Webcrawler implementiert, welcher eine Vielzahl von Restaurant-Links besucht und den Webseiteninhalt abspeichert.
Der erstellte Gold-Standard dient dazu, eine Klassifikation der Restaurantseiten anhand zwei verschiedener Ansätze durchzuführen und zu messen.
Die zwei Ansätze sind regelbasiertes Klassifizieren sowie das Klassifizieren mittels Machine-Learning.
Um die einzelnen Klassifikationen prüfen zu können, wurden in beiden Bereichen mehrere Experimente durchgeführt.
Im praktischen Teil der Arbeit wurde neben dem Webcrawler eine Webapplikation erarbeitet, welche die Suchmaschine den Benutzern zugänglich macht.
\addsec{Abstract English}
The idea of this work is to create a search engine that can be used to search menus and meals restaurant independently.
The basis of such a search engine are the websites of restaurants in combination with information such as, for example, the location.
To create a dataset of such websites, a web crawler has been implemented.
In the scientific part of this work, a gold standard was created based on this dataset.
This serves to answer the actual research question, which is the following: \\
\emph{Can Webpages of restaurants be classified with high chance of success, if they contain menu information?} \\
An F1 score of 0.8 or higher is required to answer this question with \glqq Yes\grqq{}.
Two different approaches are used for classification, in particular rule-based classification and classification with machine learning.
To be able to test these approaches, several experiments are carried out.
Finally, it is recognized that the desired F1 score was missed with both approaches.
In the practical part, in addition to the webcrawler, a web application will be developed which serves as a search engine and displays the classified data to a user.

\addchap{Danksagung}
Wir bedanken uns herzlich bei unseren Referenten, Corsin Capol und Lukas Toggenburger, dafür, dass sie uns bei dieser Arbeit viele Freiheiten gelassen und uns viele gute Ratschläge gegeben haben und uns bei wichtigen Entscheidungen unterstützten.\\
Zudem bedanken wir uns bei Norman Süsstrunk für die Hilfe bei der Erarbeitung des Webcrawlers.
\newpage

%Inhaltsverzeichnis
\tableofcontents

\clearpage 
\newpage

\addchap{Glossar}
Crawlen - Die Tätigkeit eines Webcrawlers: Das Aufrufen und Herunterladen von Webseiten\\
OSM - Opensteetmap, eine freie Geodatenbank\\
SDK - Software Development Kit (SDK) Programmierwerkzeug und Bibliothek, welches hilft, eine Software zu entwickeln\\
Seed - Eine Liste bestehend aus Webseiten, welche gecrawlt werden sollen\\
Stormcrawler - SDK zur Entwicklung eines Webcrawlers\\
Webcrawler - Programm zum automatisierten Aufrufen und Speichern einer Webseite\\
Website - Komplette Internetseite (Startseite inkl. allen Subwebsites) z.B. www.menucrawler.ch\\
Webpage - Spezifischer Teil einer Webseite z.B. www.menucrawler.ch/ueber-uns\\
idempotent - Bei gleicher Eingabe erfolgt stets die gleiche Ausgabe, egal wie oft die Aufgabe ausgeführt wird\\
Open-Source - Software, deren Quelltext frei einsehbar und änderbar ist\\
Big Data - Arbeitsumfeld, indem mit sehr vielen Datenpunkten gearbeitet wird\\
Machine-Learning - Maschinelles Lernen, der Computer wird darauf trainiert, eine Aufgabe zu verrichten\\
Pipelining - Das verbinden von mehreren Aufgaben zu einer Gesamtaufgabe\\
Stream - Kontinuierlicher Fluss von Datensätzen\\
URL - Uniform Ressource Locator - Eindeutiger Link zu einer Webpage\\
Blacklisting - Index, der zur Benachteiligung der darin aufgeführten Einträgen führt\\
Whitelisting - Index, der zur Bevorzugung der darin aufgeführten Einträgen führt\\
Labeling - Kategorisieren von Daten\\
JSON - JavaScript Object Notation, ein kompaktes Dateiformat in lesbarer Textform, geeignet für den Datenaustausch zwischen Anwendungen\\
No free lunch - Ein Theorem, welches besagt, dass es kein universelles Verfahren für die Lösung eines ML-Problems gibt\\
N-Gramme - Wörterkette mit Anzahl N Wörtern\\
Features - Merkmale, mit denen ein ML-Algorithmus trainiert werden kann\\
Feature-Extraction - Das Extrahieren von relevanten Features\\
\newpage

\numberwithin{figure}{chapter}
\pagestyle{scrheadings}
\pagenumbering{arabic}

\chapter{Einleitung}
Viele Menschen nutzen das Internet, um sich über Restaurants, deren Speisekarten und Mittagsmenüs zu informieren.
Diese Arbeit hat das Ziel, den Prototypen einer Suchmaschine zu entwickeln, über welche Restaurant-übergreifend Speisen und Menüs gesucht werden können.\\
Dazu werden verschiedene Teilkomponenten entwickelt, welche zusammen dieses Ziel erfüllen.
Bei der ersten Komponente handelt es sich um einen teilautomatisierten Webcrawler, welcher mit dem Software Development Kit \glqq StromCrawler\grqq{} erstellt wurde.
Dieser ruft Websites aus einem Seed, bestehend aus URLs von Restraunt-Websites, auf, findet deren Webpages  und speichert sie, sofern die Website dies erlaubt.
Durch diesen Webcrawler ist ein Rohdatensatz entstanden, der aus ca. 70'000 Webpages besteht.\\
Diese Rohdaten werden mittels einer weiteren Komponente, dem Classifier, klassifiziert, ob es sich dabei um Menüseiten bzw. Speisekarten handelt.
Diese Aufgabe wird mit zwei verschiedenen Ansätzen in einem Experiment versucht zu lösen.
Der erste Ansatz basiert auf regelbasierter Klassifikation.
Dabei wurden verschiedene Regeln erstellt, welche die Webpages Anhand des Textes und der Titels einer Webpage klassifizieren.
In einem zweiten Ansatz wird diese Aufgabe mit Methoden des maschinellen Lernens (Machine Learning) sowie der Verarbeitung natürlicher Sprache (NLP) versucht zu lösen.
Um die Algorithmen des maschinellen Lernens trainieren und die Ergebnisse der Klassifikation messen zu können, ist ein Gold Standard mit knapp 7000 von Hand kategorisierten Webpages erstellt worden.\\
Die Rohdaten werden mit dem Algorithmus, der die besten Ergebnisse liefert, klassifiziert.
Diejenigen Webpages, welche als Menüseite oder Speisekarte klassifiziert wurden, werden in einer Search Engine gespeichert und verwaltet.
Diese Komponente wird mit der Software ElasticSearch umgesetzt.
Die darin verwalteten Daten müssen zuvor auf einen einheitlichen Stand gebracht werden, damit jeder Eintrag Informationen über das Restaurant, die URLs der Menüseiten sowie der Geolocation enthält.\\
Als letzte Komponente kommt eine Webapplikation zum Einsatz. 
Mit dieser kann nach einer Speise gesucht werden und diese Suche optional mittels Umkreisfilter eingrenzen.
Dadurch lassen sich Restaurants, die eine spezifische Speise anbieten und in einem gewünschten Umkreis liegen, auf einer Karte anzeigen.
Die URL der Menüseite wird dabei verlinkt, sodass ein Benutzer diese direkt aufrufen kann.\\
Diese Arbeit beinhaltet zwei verschiedene Teile, einen wissenschaftlichen und einen praktischen Teil.
Im wissenschaftlichen Teil wird ein Gold Standard sowie die Klassifikation von dessen methodisch erarbeitet.
Im praktischen Teil werden die klassifizierten Webpages über eine Webapplikation verfügbar gemacht.
\newpage

\chapter{Stand der Forschung}
\section{Gold Standard}
% Was definiert einen guten Gold Standard?
Öffentliche Datensätze, welche Daten in Textform beinhalten, sind in verschiedenen Kategorien verfügbar\footnote{\url{https://medium.com/@dataturks/rare-text-classification-open-datasets-9d340c8c508e}}.
Auch Datensätze, die Informationen über Menüs beinhalten, sind bereits öffentlich verfügbar \footnote{\url{https://data.world/data-society/discover-the-menu}}.
Deutsche Textdatensätze sind bereits weniger verbreitet wie solche in englischer Sprache, jedoch sind ebenfalls welche frei verfügbar\footnote{\url{http://wortschatz.uni-leipzig.de/en/download/}}.
Einen Datensatz, welcher auf der deutschen Sprache basiert und Informationen über Menüseiten beinhaltet, konnte jedoch nicht gefunden werden, daher ist die Erarbeitung eines solchen Datensatzes Teil dieser Arbeit.
\section{Klassifikation}
Eine Methode, um Text in verschiedene Kategorien zu klassifizieren, ist das Erstellen von Regeln.
Mit einem ausgefeilten Regelsatz kann dies bei einer geringen Anzahl verschiedener Kategorien gut funktionieren \cite[p.125]{jackson2007natural}.
Das Erstellen eines solchen Regelsatzes ist zeitaufwendig und muss mit einem repräsentativen Datensatz überprüft werden \cite[p.125]{jackson2007natural}.
Somit ist es ein iterativer Prozess und der Ersteller muss sich im Themengebiet gut auskennen \cite[p.125]{jackson2007natural}.
Solche Regelsätze bestehen aus \glqq Wenn-Dann\grqq-Bedingungen, welche mit logischen \glqq UND\grqq{} bzw. \glqq ODER\grqq{} Operatoren verknüpft sind in Kombination mit regulären Ausdrücken \cite[p.126]{jackson2007natural}.
Mit solchen Regelsätzen könnten zwar hohe Scores erreicht werden, ein grosses Problem des Erstellens solcher Regelsätze ist der Zeitaufwand, denn dieser steigt linear mit der Komplexität \cite[p.127]{jackson2007natural}.
Aus diesem Grund besteht ein hoher Bedarf, Daten mittels statistischer Methoden automatisiert zu klassifizieren, ohne dass ein Mensch zuerst neue Regelsätze implementieren muss \cite[p.127]{jackson2007natural}.\\
An diesem Punkt kommen Machine-Learning-Algorithmen zum Zug.
Diese analysieren einen Datensatz, bei dem die Proben bereits eine Zuteilung zur jeweiligen Kategorie hinterlegt haben, und klassifiziert künftige Daten anhand dieser Analyse \cite[p.127]{jackson2007natural}. 
\section{No-Free-Lunch Theorem}\label{sec:nofreelunch}
Ein Theorem vom bekannten Informatiker David H. Wolpert, welches sinngemäss die Aussage trifft, dass kein Klassifizierer existiert, welcher für eine Vielzahl von Klassifikationsproblemen geeignet wäre\cite[p.]{Wolpert1996TheLO}.

\section{Klassifikationsalgorithmen}	% stand der forschung evt. classification von web pages?
\begin{itemize}
	\item Naive Bayes Modelle
	\begin{itemize}
		\item Bernoulli-Naive-Bayes
		\item Complement-Naive-Bayes
		\item Multinomial-Naive-Bayes
		\item Gaussian-Naive-Bayes
	\end{itemize}
	\item Nearest Neighbor Modelle
	\begin{itemize}
		\item KNeighborClassifier
		\item Nearest Centroid
	\end{itemize}
\end{itemize}
\subsection{Lineare Modell}
Lineare Modelle sind in der Annahme, dass ein linearer Zusammenhang zwischen Eingangsvariablen und Ausgangsvariablen besteht.
Lineare Modelle versuchen Parameter zu einer linearen Gleichung zu finden, welche die Trainingsdatenpunkte optimal abdeckt.
Das optimale Abdecken wird mit einer Loss-Funktion ermittelt.
Die Loss-Funktion berechnet den Unterschied zwischen vorhergesagtem Wert zu tatsächlichem Wert des Trainingsdatenpunktes.
Lineare Modelle verwenden Optimierungsverfahren, welche iterativ Parameter verändern, um die Werte der Loss-Funktion zu minimieren und somit die bestmögliche Parameterzusammensetzung zu ermitteln.
\subsubsection{Ridge Classifier}
Ridge Classifier ist ein linearer Klassifizierer, welcher als Loss-Funktion die \glqq Least Square\grqq{} Funktion (eine Minimierung der quadrierten Abweichungen) und als Regularisierung die L2-Norm verwendet. Die L2-Norm wird pythagoreisch aus den Vektorwerten berechnet\footnote{\url{https://machinelearningmastery.com/vector-norms-machine-learning/} abgerufen am: 13.07.2019}.
Die Kombination aus oben erwähnter Loss-Funktion und der L2-Norm wird auch \glqq Ridge Regression\grqq{} genannt, von welcher der Klassifizierer auch seine Bezeichnung erhielt\footnote{\url{https://scikit-learn.org/stable/modules/generated/sklearn.linear_model.Ridge.html} abgerufen am: 13.07.2019}\cite{scikit-learn}.
\subsubsection{Passive Agressive Classifier}
Der PassiveAggressive Classifier ist ein Klassifizierer für grosse Datenmengen.
Er ist verwandt mit dem Perceptron Algorithmus, da er keine Lernrate benötigt.
Der Klassifizierer funktioniert grob umschrieben so, dass wenn die Klassifizierung korrekt ist, er sich passiv verhält und seine internen Gewichte nicht anpasst.
Erst bei einer falschen Klassifizierung ändert seine interne Gewichte aggressiv, so dass die Fehlklassifizierung behoben wird\cite{crammer2006online}.
\subsubsection{SGDClassifier}
SGDClassifier gehört zur Familie der lineare Modelle\footnote{\url{https://scikit-learn.org/stable/modules/generated/sklearn.linear_model.SGDClassifier.html} abgerufen am: 14.05.2019}\cite{scikit-learn}.\\
Der \glqq Stochastic-Gradient-Descent-Classifier\grqq{} verwendet als Optimierungsvefahren das \glqq Stochastic Gradient Descent\grqq{} Verfahren\footnote{\url{https://scikit-learn.org/stable/modules/sgd.html} abgerufen am: 14.05.2019}.
Bei diesem Verfahren kommt der mathematische Gradient zum Einsatz.
Der Gradient zeigt bei Anpassungen der Parameter, immer in die Richtung mit der grössten Änderung des Outputs.
Der Output wäre in diesem Falle die Aufzeichnung des Loss-Funktion.\\
Der stochastische/probabilistische Anteil dieses Verfahrens bedeutet, dass bei der Optimierung jeweils nur ein Parameter zufällig ausgesucht und angepasst wird und seine Auswirkung anschliessend ermittelt wird.
Dies hat den Vorteil, dass die Trainingszeit optimiert werden kann und zugleich gute Werte erzielt werden können.\\
In der Abbildung \cref{fig:sgd} wird der Verlauf der Loss-Funktion aufgezeigt und wie der Gradient zu interpretieren ist.
\begin{figure}[H]	
	\includegraphics[width=1\columnwidth,keepaspectratio]{img/sgd.png}
	\caption{Visualisierung des Gradienten und der Loss-Funktion (J(w) = Loss; w = Parameteränderung)}
	\source{\url{http://rasbt.github.io/mlxtend/user_guide/general_concepts/gradient-optimization/}}
	\label{fig:sgd}
\end{figure}
\subsubsection{Perceptron}
Der Perceptron-Algorithmus basiert auf dem SGDClassifier\footnote{\url{https://scikit-learn.org/stable/modules/generated/sklearn.linear_model.Perceptron.html} abgerufen am: 14.05.2019}\cite{scikit-learn}.
Hinter der Perceptron-Implementierung steckt lediglich ein SGDClassifier mit spezieller Konfiguration.\\
Die Perceptron-Implementierung benutzt in der Konfiguration die Loss-Funktion \glqq perceptron\grqq{}, keine Strafe bei falscher Klassifizierung und eine konstante Lernrate.
\subsection{Naive Bayes}
\subsection{DecisionTree}\label{sec:trees}
Der DecisionTree-Algorithmus baut schrittweise eine Baumstruktur von Entscheidungszweige auf, um eine Klassifizierungsaufgabe zu meistern.
DecisionTrees versuchen eine komplexe Aufgabe in Teilprobleme zu zerlegen und diese mit einfachen Entscheidungen zu bewältigen.
DecisionTrees-Strukturen können verbessert werden, indem die Tiefe der Äste oder die Anzahl der Äste angepasst werden kann.
Bei stetiger Erhöhung der Tiefe oder der Anzahl der Äste, steigt auch die Zeitkomplexität der DecisionTrees\cite{safavian1991survey}.
\subsection{Ensemble-Learning}
Ensemble-Learning ist ein Zusammenschluss von mehreren unterschiedlichen Klassifizierern, welche mit einem Voting-Verfahren eine schlussendliche Klassifizierung durchführen.
Ensemble-Learning verfolgt die Annahme, dass mehrere Algorithmen im Plenum eine bessere Aussage abliefern können, als einen Algorithmus alleine\cite{freund1999short}.
\subsubsection{RandomForestClassifier}
RandomForest gehört ebenfalls zur Familie der Ensemble-Learner.\footnote{\url{https://scikit-learn.org/stable/modules/generated/sklearn.ensemble.RandomForestClassifier.html} abgerufen am: 14.05.2019}
RandomForest, wie der Name schon sagt, ist eine Zusammensetzung von einer Vielzahl von unterschiedlichen DecisionTrees \cref{sec:trees}.\\
RandomForest verwendet nun eine Vielzahl von DecisionTrees, die alle unterschiedliche Tiefen oder Anzahl Äste besitzen.
Dadurch können Entscheidungsausreisser aufgefangen und durch den Mehrheitsentscheid gedämpft werden\cite{liaw2002classification}.
\subsubsection{AdaBoostClassifier}
Bei vielen Ensemble-Verfahren werden alle Klassifizierer parallel trainiert und geben ihr Votum gleichzeitig ab.\\
Adaboost verwendet jedoch die Methode des \glqq Boosting\grqq{}, welche Ähnlichkeit mit der Theorie der genetischen Algorithmen hat\footnote{\url{https://scikit-learn.org/stable/modules/ensemble.html} abgerufen am: 14.05.2019}.
Bei Adaboost wird ein Algorithmus trainiert, validiert und als Ursprung verwendet. Alle zusätzlichen Algorithmen, welche das finale Voting durchführen, werden vom Ursprungsalgorithmus abgeleitet.
Es werden jedoch bei den Abkömmlingen die internen Parameter schrittweise verbessert und versucht die Fehler des \glqq Vater-Algorithmus\grqq{} zu vermeiden.
AdaBoost kann verbessert werden, indem die Anzahl von Vererbungsschritten angepasst wird\cite{freund1999short}.
\subsection{Nearest Neighbor Modelle}
Der Nearest Neigbhor (NN) Algorithmus ist einer der simpelsten Entscheidungsprozesse für Klassifikationen.
Beim NN Algorithmus werden Datenpunkte entsprechend ihren nächsten Nachbarn klassifiziert\cite{cover1967nearest}.
Dieses Verhalten ist in der \cref{fig:knn} für unterschiedliche Schwellwerte der Anzahl Nachbarn ersichtlich.
\begin{figure}[H]	
	\includegraphics[width=0.7\columnwidth,keepaspectratio]{img/knn.png}
	\caption{Darstellung einer simplen K-Nearest Neighbor Prozedur.}
	\source{\cite{cover1967nearest}}
	\label{fig:knn}
\end{figure}
\subsubsection{K-Nearest-Neighbor}
Der KNN Algorithmus funktioniert nach dem gleichen Prinzip wie der NN Algorithmus.
Ein Unterschied ist, dass eine spezifische Anzahl von Nachbarn mit der Kennzahl K definiert wird.
Alle K-nächsten Datenpunkten klassifizieren den gesuchten Datenpunkt mit einem Mehrheitsentscheid.
Dies ist in der \cref{fig:knn} für K=3 und K=5 ersichtlich.
K kann frei gewählt werden und dient hervorragend als Parameter für eine Hyperparametersuche \cref{sec:hyp}\cite{cover1967nearest}.
\subsubsection{Nearest Centroid}
Der Nearest Centroid Algorithmus basiert auf dem Prinzip des NN Algorithmus.
Zuerst werden die Schwerpunkte aller Klassen berechnet, indem alle zur Klasse zugewiesenen Datenpunkte miteinbezogen werden.
Danach wird der zu klassifizierende Datenpunkt der Klasse zugewiesen, mit der kleinsten Differenz von Datenpunkt zu Schwerpunkt der Klasse.
Somit wird für den Nearest Centroid Algorithmus nicht der Mehrheitsentscheid der nächsten Nachbarn zur Klassifikation verwendet, sondern die Distanzen zu den Klassenschwerpunkten.
Dieses Verhalten ist in \cref{fig:centroid} ersichtlich.
Wobei s1 die Distanz vom Schwerpunkt der Klasse \glqq Investing Picks\grq{} zum Datenpunkt und s2, beziehungsweise s3, zu den Schwerpunkten ihrer dazugehörigen Klassen.
\begin{figure}[H]	
	\includegraphics[width=0.7\columnwidth,keepaspectratio]{img/centroid.png}
	\caption{Darstellung einer simplen Nearest Centroid Prozedur.}
	\source{\url{https://www.slideshare.net/mgrcar/text-and-text-stream-mining-tutorial-15137759}}
	\label{fig:centroid}
\end{figure}
\subsection{Support Vector Machine Modelle}
SVM-Classifier (Support Vector Machine) erzielen gute Resultate bei der Klassifizierung von Textdateien.
Ein Vorteil von SVM ist, dass ihre Lernrate unabhängig von der Dimension der Features ist\cite{joachims1998text}.
Dies wird erreicht, da SVM nicht auf die ganzen Features angewiesen ist.
Zwischen den unterschiedlichen Klassen werden Hyperebenen gelegt, welche das Ziel haben, die Abstände zu den einzelnen Klassen zu maximieren.
In der \cref{fig:svm} ist eine solche Hyperebene ersichtlich.
Die Features, welche am nächsten zu den Hyperebenen liegen, werden Support Vektoren genannt.
Der Algorithmus kann nach der Trennung der Klassen nun die Position von neuen Features bestimmen und somit eine Schätzung einer geeigneten Klasse vornehmen\cite{tong2001support}.
\begin{figure}[H]	
	\includegraphics[width=0.7\columnwidth,keepaspectratio]{img/svm.png}
	\caption{Darstellung einer simplen Support Vector Machine.}
	\source{\cite{tong2001support}}
	\label{fig:svm}
\end{figure}
\section{Hyperparametertuning}\label{sec:hyp}
Parameter, welche nicht selbstständig vom Machine-Learning Modell angepasst werden können, werden Hyperparameter genannt und sind meist Übergabeparameter in der Konstruktor-Methode.
Diese Hyperparameter definieren das Verhalten von den Modellen.
Die geeignetsten Hyperparameter können mit einer ausführlichen Suche gefunden werden, was als Hyperparametertuning bezeichnet wird\footnote{\url{https://scikit-learn.org/stable/modules/grid_search.html} abgerufen am: 14.05.2019}\cite{scikit-learn}.
\newpage

\chapter{Forschungsfrage und Methodik}
\section{Forschungsfrage}
Aus dem Stand der Forschung ist die folgende Forschungsfrage entstanden:\\

\emph{Können Restaurant-Webseiten mit hoher Wahrscheinlichkeit klassifiziert werden, ob sie Menüinformationen beinhalten?}\\

Als hohe Erfolgschance wird ein F1-Score von mindestens 0.8 definiert.
Es handelt sich um eine binäre Klassifikation, also eine Einteilung in zwei Kategorien, nämlich \glqq Menüseite\grqq{} oder \glqq Keine Menüseite\grqq.
Zudem sind die unter \cref{chap:scope} definierten Einschränkungen zu berücksichtigen.
\section{Methodik}
Um die Forschungsfrage beantworten zu können, werden zwei verschiedene Ansätze, namentlich das regelbasierte Klassifizieren sowie die Klassifikation mittels Machine-Learning angewendet.
Dies findet in mehreren Experimenten statt.
Um die Ergebnisse dieser Experimente messen und miteinander vergleichen zu können, wird ein Gold Standard erstellt.
\subsection{Gold Standard}
Als Startpunkt zur Erstellung des Gold Standards dient der Output des in \cref{chap:engineering} erstellten Webcrawlers.
Diese Daten werden von Hand kategorisiert, sodass ein representativer Datensatz zur Messung der Klassifikation entsteht.
Um diese Daten kategorisieren zu können, müssen die Kategorien definiert und ein Entscheidungsraster erstellt werden.
Die zu klassifizierenden Daten werden zufällig ausgewählt, sodass eine natürliche Verteilung der jeweiligen Kategorien entsteht.
Die Daten werden zum Schluss in einen Trainings- und Testdatensatz unterteilt.
\subsection{Klassifikation}
Um die Daten des Gold Standards zu klassifizieren, werden sie zuerst mittels Preprocessing in eine standardisierte Form gebracht.
Die Klassifikation selbst wird in zwei verschiedene Kategorien unterteilt, das regelbasierte Klassifizieren sowie die Klassifikation mittels Machine-Learning.
Beim regelbasierten Klassifizieren werden fünf Methoden entwickelt, anhand welcher die Daten klassifiziert werden.
Bei der Klassifikation mittels Machine Learning werden verschiedene Algorithmen verwendet.
Um einen möglichst hohen F1-Score zu erreichen, wird in beiden Fällen mit verschiedenen Konfigurationen gearbeitet.\\
Die Experimente werden bei jeglichen Klassifikationsmethoden anhand des Trainingsdatensatzes durchgeführt.
Um die schlussendlich gültigen Resultate zu erheben, wird bis dahin unberührte Testdatensatz verwendet.
Die Ergebnisse werden anhand des F1-Scores gemessen.
Wenn zwei verschiedene Konfigurationen denselben F1-Score erreichen, wird diejenige Konfiguration bevorzugt, welche die höhere Precision erreicht.

\newpage

\chapter{Gold Standard}
\section{Beschreibung des Gold Standards}
Der erarbeitete Gold Standard besteht insgesamt aus 6963 Dateien des Formats JSON, welche jeweils eine Webpage einer Restaurant-Website repräsentieren.
Diese Daten sind in einen Test- und einen Trainings- und Validierungsdatensatz unterteilt.
Der Gold Standard ist in die folgenden drei Kategorien unterteilt:\\

\begin{forest}
	for tree={
		font=\ttfamily,
		grow'=0,
		child anchor=west,
		parent anchor=south,
		anchor=west,
		calign=first,
		edge path={
			\noexpand\path [draw, \forestoption{edge}]
			(!u.south west) +(7.5pt,0) |- node[fill,inner sep=1.25pt] {} (.child anchor)\forestoption{edge label};
		},
		before typesetting nodes={
			if n=1
			{insert before={[,phantom]}}
			{}
		},
		fit=band,
		before computing xy={l=15pt},
	}
	[Gold Standard (6963 Dateien)
	[test (100 Dateien)
	[neg (50 Dateien)]
	[pos\textunderscore daily\textunderscore menu (10 Dateien)]
	[pos\textunderscore menu (40 Dateien)]
	]
	[train-validation (6863 Dateien)
	[neg (6257 Dateien)]
	[pos\textunderscore daily\textunderscore menu (87 Dateien)]
	[pos\textunderscore menu (519 Dateien)]
	]
	]
\end{forest}\\

%\begin{tabular}{|l|l|l|l|}
%	\hline
%	Kategorie & Ordnername & Anzahl Dateien & Anzahl Dateien im Testsatz\\
%	\hline
%	Menü & pos\textunderscore menu & 519 & 40 \\
%	Tagesmenü & pos\textunderscore daily\textunderscore menu & 87 & 10 \\
%	Kein Menü & neg & 6257 & 50\\
%	\hline
%\end{tabular}\\
Jeder Eintrag dieses Gold Standards beinhaltet die folgenden Informationen:
\begin{itemize}
	\item \glqq date \grqq  - Zeitpunkt, zu welchem die Webpage aufgerufen wurde
	\item \glqq text \grqq  - Vom Webcrawler extrahierter Text, welcher die Webpage beinhaltet
	\item \glqq encoding \grqq  - Das von der Webpage verwendete Encoding
	\item \glqq title \grqq  - Inhalt des gleichnamigen HTML-Metatags
	\item \glqq url \grqq  - URL der Webpage
	\item \glqq content \grqq  - Der statische HTML-Inhalt der Webpage	
\end{itemize}
\section{Entscheidungsraster}
Der Gold Standard wurde anhand des Entscheidungsrasters erstellt, welches in der Abbildung \cref{fig:classificationtree} dargestellt wird.
\begin{figure}[H]	
	\includegraphics[width=1\columnwidth,keepaspectratio]{img/man-classification-tree.jpg}
	\caption{Entscheidungsraster}
	\label{fig:classificationtree}
\end{figure}
Obwohl bereits beim Webcrawler eine Spracherkennung eingesetzt wurde, damit nur als deutsch erkannte Webpages gespeichert werden, wurde bei der manuellen Klassifikation nochmals darauf geachtet, dass die Webpage in deutsch verfasst wurde.
Dabei ist anzumerken, dass gewisse Begriffe, vor allem für Speisebezeichnungen, auch fremdsprachig sein dürfen, da Speisebezeichnungen je nach Küche international ausgelegt sind.
Der Anbieter muss zwingend ein Restaurant, Take-Away oder Lieferdienst sein.
Der Inhalt muss als statisches HTML verfügbar sein, da der Webcrawler nicht mit dynamisch gerenderten Websites umgehen kann.
Der Name der Speise und eine genauere Beschreibung oder der Preis muss vorhanden sein.
Danach folgt die Unterscheidung zwischen zeitlich begrenzten und unbegrenzten Angeboten, welche zur Kategorisierung führt.
Getränkekarten wurden explizit negativ klassifiziert.
\section{Seed}
Das Seed wurde aus den folgenden zwei Quellen zusammengestellt:
\begin{itemize}
	\item OpenStreetMap - 3557 URLs
	\item Lunch-Check - 3803 URls
\end{itemize}
Diese URLs wurden zusammengeführt, zudem sind Duplikate entfernt worden.
Daraus ist ein Seed entstanden, welches 5870 Einträge von Restaurant-URLs enthält.
Dabei wurden aus den nun aufgeführten Gründen mehrere Einträge entfernt:
\begin{itemize}
	\item Die Website enthält mehr als 300 Webpages
	\item Die Website ist offensichtlich keine Restaurant-Website
\end{itemize}
Websites mit mehr als 300 Webpages wurden entfernt, da diese keine typische Restaurant-Website repräsentieren und somit das Gesamtbild verzerren.
Es kann keine Gewähr gegeben werden, dass dieses Seed nur Restaurant-Websites beinhaltet, da nicht jeder Eintrag geprüft wurde.

\newpage

\chapter{Experiment und Klassifizierung}
\label{cap:exp_class}
Sämtliche Klassifikation findet anhand der Informationen der Attribute \glqq text \grqq und \glqq title \grqq des Gold Standards statt.
Zudem werden alle Webpages der Kategorie \glqq Tagesmenü \grqq der Kategorie \glqq Menü \grqq hinzugefügt, da bei dieser Klassifikation nur zwischen \glqq Menü \grqq oder \glqq Kein Menü \grqq unterschieden wird.
\section{Regelbasiertes Klassifizieren}
Jedes Regelset wurde anhand des Goldstandards getestet und evaluiert.
Verschiedene Parameterwerte und Kombinationen wurden getestet, um möglichst hohe Werte der gemessenen Metriken zu erreichen.
Bei allen Regelsets sind alle Methoden des Preprocessings aktiv gewesen. 
\subsection{Regelset: Menü im Titel}
Für dieses Regelset sind keine Parameter verfügbar, daher ist nur eine Konfiguration durchgführt worden.
Diese hat folgende Metriken ergeben:\\
\begin{table}[H]
\caption{Score des Regelsets: Menü im Titel}
\centering
\begin{tabular}{|l|l|l|}
	\hline
	F1-Score & Precision & Recall\\
	\hline
	0.17 & 0.43 & 0.11  \\
	\hline
\end{tabular}
\end{table}
\subsection{Regelset: Preisdetektor}
Durch die Konfiguration kann ein Schwellwert für die Anzahl erkannter Preise angegeben werden, die vorhanden sein müssen, um eine Webpage als positiv zu klassifizieren.\\
\begin{table}[H]
\caption{Scores des Regelsets: Preisdetektor}
\centering
\begin{tabular}{|l|l|l|l|}
	\hline
	Schwellwert & F1-Score & Precision & Recall\\
	\hline
	1 & 0.45 & 0.36 & 0.60  \\
	2 & 0.46 & 0.45 & 0.47 \\
	3 & 0.41 & 0.49 & 0.36 \\
	\hline
\end{tabular}
\end{table}
Das beste Ergebnis hat ein Schwellwert von zwei erzielt, danach ist der F1-Score wieder schlechter geworden.
Daraus wurde geschlussfolgert, dass für dieses Regelset das Maximum bereits erreicht wurde.
\subsection{Regelset: Kombination aus Menü im Titel und Preisdetektor}
Bei dieser Konfiguration ebenfalls kann der Schwellwert für die Anzahl erkannter Preise angegeben werden.\\
\begin{table}[H]
	\caption{Scores des Regelsets: Kombination aus Menü im Titel und Preisdetektor}
	\centering
\begin{tabular}{|l|l|l|l|}
	\hline
	Schwellwert & F1-Score & Precision & Recall\\
	\hline
	1 & 0.45 & 0.35 & 0.65 \\
	2 & 0.47 & 0.43 & 0.52 \\
	3 & 0.43 & 0.45 & 0.41 \\
	\hline
\end{tabular}
\end{table}
Auch bei diesem Regelset hat die Konfiguration mit einem Schwellwert von zwei das beste Ergebnis erzielt.
\subsection{Regelset: Listing}
Beim Listing können zwei Schwellwerte angeben werden, einen für die Anzahl übereinstimmender Wörter aus der Whitelist und einen für die Blacklist.
In einem ersten Versuch wurden identische Schwellwerte gewählt und jeweils erhöht:\\
\begin{table}[H]
	\caption{Scores der ersten Iteration des Regelsets: Listing}
	\centering
\begin{tabular}{|l|l|l|l|l|}
	\hline
	Schwellwert Whitelist & Schwellwert Blacklist & F1-Score & Precision & Recall\\
	\hline
	1 & 1 & 0.17 & 0.32 & 0.11 \\
	5 & 5 & 0.39 & 0.62 & 0.29 \\
	10 & 10 & 0.42 & 0.77 & 0.29 \\
	20 & 20 & 0.33 & 0.85 & 0.21 \\
	30 & 30 & 0.27 & 0.91 & 0.16 \\
	\hline
\end{tabular}
\end{table}
Dieser Versuch hat gezeigt, dass ein maximaler F1-Score bei gleichen Werten zwischen 5 und 20 zu erreichen ist.\\
Da gleich gewählte Werte keine zufriedenstellende Ergebnisse erzielten, wurden in zweiten Versuch wurden unterschiedliche Verhältnisse getestet:\\
\begin{table}[H]
	\caption{Scores der zweiten Iteration des Regelsets: Listing}
	\centering
\begin{tabular}{|l|l|l|l|l|}
	\hline
	Schwellwert Whitelist & Schwellwert Blacklist & F1-Score & Precision & Recall\\
	\hline
	5 & 1 & 0.12 & 0.82 & 0.07 \\
	1 & 5 & 0.31 & 0.24 & 0.45 \\
	\hline
\end{tabular}
\end{table}
Aus diesem Versuch entstand die Schlussfolgerung, dass ein höherer Schwellwert der Blacklist als der Whitelist erforderlich ist, um einen möglichst hohen Score zu erreichen.
Diese Erkenntnis wurde in einem weiteren Versuch in mehreren Iterationen getestet:\\
\begin{table}[H]
	\caption{Scores der dritten Iteration des Regelsets: Listing}
	\centering
\begin{tabular}{|l|l|l|l|l|}
	\hline
	Schwellwert Whitelist & Schwellwert Blacklist & F1-Score & Precision & Recall\\
	\hline
	2 & 10 & 0.42 & 0.33 & 0.61 \\
	3 & 15 & 0.52 & 0.42 & 0.70 \\
	3 & 20 & 0.51 & 0.39 & 0.75 \\
	4 & 15 & 0.53 & 0.47 & 0.61 \\
	5 & 15 & 0.55 & 0.54 & 0.56 \\
	5 & 20 & 0.54 & 0.50 & 0.60 \\
	5 & 21 & 0.54 & 0.49 & 0.60 \\
	6 & 20 & 0.55 & 0.56 & 0.55 \\
	7 & 20 & 0.55 & 0.60 & 0.50 \\
	8 & 20 & 0.53 & 0.64 & 0.46 \\
	\hline
\end{tabular}
\end{table}
Die Schwellwerte wurden stetig erhöht.
Verschiedene Schwellwerte in unterschiedlichen Verhältnissen wurden dabei getestet.
Sobald einer dieser Schwellwerte zu einem schlechteren Ergebnis geführt hat, wurde er wieder reduziert oder ein neuer Verhältnis wurde getestet.
Beim Verhältnis 6/20 bzw. 7/20 wurde das Maximum des F1-Scores erreicht.
Da diese Werte nicht im Bereich einer brauchbaren Klassifikation sind, wurde auf das Ermitteln aller möglichen Kombinationen verzichtet.
\subsection{Regelset: Bag of Words}
Bei diesem Regelset kann die Grösse der Black- und Whitelist (Features), das Verhältnis zwischen Test- und Trainingsdaten (Split) sowie ein Schwellwert angegeben werden.
Für das Verhältnis zwischen Test- und Trainingsdaten wurden die Werte 0.3, 0.5 und 0.7 getestet.
In einer ersten Iteration wurde die Anzahl von 200 Features und ein Split von 0.3 verwendet, um herauszufinden, ob ein positiver oder negativer Schwellwert bessere Werte erzielt.\\
\begin{table}[H]
	\caption{Scores der ersten Iteration des Regelsets: Bag of Words}
	\centering
\begin{tabular}{|l|l|l|l|}
	\hline
	Schwellwert & F1-Score & Precision & Recall\\
	\hline
	0 & 0.64 & 0.58 & 0.71 \\
	2 & 0.51 & 0.38 & 0.75 \\
	-2 & 0.68 & 0.74 & 0.63 \\
	\hline
\end{tabular}
\end{table}
Dabei wurde erkannt, dass sich ein negativer Schwellwert positiv auf den F1-Score auswirkt.\\
In der zweiten Iteration wurde der Schwellwert weiter verkleinert.\\
\begin{table}[H]
	\caption{Scores der zweiten Iteration des Regelsets: Bag of Words}
	\centering
\begin{tabular}{|l|l|l|l|}
	\hline
	Schwellwert & F1-Score & Precision & Recall\\
	\hline
	-3 & 0.66 & 0.80 & 0.57 \\
	-4 & 0.66 & 0.86 & 0.54 \\
	-5 & 0.66 & 0.89 & 0.52 \\
	\hline
\end{tabular}
\end{table}\
Da diese Werte sich fast nicht unterscheiden, wurden alle weiterverwendet, um einen maximalen Score zu evaluieren.
In einer dritten, ausführlicheren Iteration sind die Anzahl Features von 200 - 400 sowie die drei oben genannten Verhältnisse zusammen mit den vier Schwellwerten getestet worden. Die folgende Tabelle zeigt diese Tests, sortiert nach bestem F1-Score:\\
\begin{table}[H]
	\caption{Scores der dritten Iteration des Regelsets: Bag of Words}
	\centering
\begin{tabular}{ | l | l | l | l | l | l | }
	\hline
	Split & Features & Limit & F1 & Pre & Rec \\ \hline
	0.3 & 400 & -3 & 0.72 & 0.74 & 0.7 \\ 
	0.3 & 400 & -4 & 0.72 & 0.79 & 0.66 \\
	0.3 & 400 & -5 & 0.72 & 0.81 & 0.64 \\
	0.7 & 400 & -3 & 0.72 & 0.75 & 0.69 \\
	0.7 & 400 & -4 & 0.72 & 0.79 & 0.66 \\
	0.5 & 300 & -3 & 0.71 & 0.79 & 0.65 \\
	0.5 & 400 & -3 & 0.70 & 0.74 & 0.67 \\
	0.5 & 400 & -4 & 0.70 & 0.79 & 0.63 \\ 
	0.7 & 400 & -2 & 0.70 & 0.70 & 0.70 \\ 
	0.7 & 300 & -3 & 0.70 & 0.74 & 0.67 \\
	0.7 & 300 & -4 & 0.70 & 0.80 & 0.63 \\
	0.7 & 300 & -5 & 0.70 & 0.84 & 0.60 \\
	0.7 & 400 & -5 & 0.70 & 0.82 & 0.62 \\ 
	0.3 & 400 & -2 & 0.69 & 0.66 & 0.73 \\ 
	0.5 & 400 & -2 & 0.69 & 0.69 & 0.69 \\ 
	0.5 & 400 & -5 & 0.69 & 0.82 & 0.60 \\ 
	0.7 & 200 & -3 & 0.69 & 0.76 & 0.63 \\ 
	0.7 & 200 & -4 & 0.69 & 0.81 & 0.60 \\ 
	0.3 & 300 & -2 & 0.68 & 0.73 & 0.64 \\ 
	0.3 & 300 & -3 & 0.68 & 0.78 & 0.60 \\ 
	0.3 & 200 & -2 & 0.68 & 0.74 & 0.63 \\ 
	0.5 & 200 & -3 & 0.68 & 0.77 & 0.60 \\ 
	0.5 & 300 & -4 & 0.68 & 0.81 & 0.59 \\ 
	0.7 & 300 & -2 & 0.68 & 0.69 & 0.68 \\ 
	0.5 & 200 & -2 & 0.67 & 0.71 & 0.64 \\ 
	0.3 & 300 & -4 & 0.66 & 0.82 & 0.56 \\ 
	0.3 & 200 & -3 & 0.66 & 0.80 & 0.57 \\
	0.3 & 200 & -4 & 0.66 & 0.86 & 0.54 \\
	0.3 & 200 & -5 & 0.66 & 0.89 & 0.52 \\ 
	0.5 & 200 & -4 & 0.66 & 0.81 & 0.56 \\
	0.5 & 300 & -5 & 0.66 & 0.83 & 0.55 \\ 
	0.7 & 200 & -2 & 0.66 & 0.66 & 0.66 \\ 
	0.7 & 200 & -5 & 0.66 & 0.85 & 0.55 \\ 
	0.3 & 300 & -5 & 0.65 & 0.85 & 0.52 \\ 
	0.5 & 200 & -5 & 0.65 & 0.87 & 0.52 \\
	0.5 & 300 & -2 & 0.46 & 0.62 & 0.36 \\ \hline
\end{tabular}
\end{table}
Daraus lässt sich schliessen, dass eine hohe Anzahl Features zu einem besseren Ergebnis führt.
Das Verhältnis zwischen Test- und Trainingsdaten ist nicht so relevant, da sowohl das Verhältnis 0.3 als auch 0.7 zu hohen Scores führt.
Der Schwellwert ist im Bereich -2 bis -5 ebenfalls nicht aussagekräftig, da auch dieser bei den besten Scores vertreten ist.
Es muss zudem berücksichtigt werden, dass der Split zufällig gewählt wird und keine Kreuzvalidierung stattfindet, dadurch können diese Ergebnisse variieren.
\section{Auswirkungen des Preprocessings}
\subsection{Regelbasiertes Klassifizieren}
Bei den Methoden des regelbasierten Klassifizierens trägt das Preprocessing einen erheblichen Teil zum Erfolg bei.
Die Methode \glqq Menü im Titel\grqq{} profitiert davon, dass Umlaute mit den entsprechenden Selbstlauten ersetzt werden.
Der Preisdetektor funktioniert ohne den gleichnamigen Preprocessing-Schritt gar nicht, da nach dem Ersatzwort gesucht wird.
Beide Punkte gelten auch für die Kombination dieser Methoden.\\
Das Listing profitiert von mehreren Preprocessing-Schritten.
Das Ersetzen der Grossbuchstaben durch Kleinbuchstaben, der Preis- und Getränkedetektor sowie die Stammformreduktion führen dazu, dass im Text vorkommende Worte den Worten der jeweiligen Listen besser zugeordnet werden können.
Die Methode \glqq Bag of Words\grqq{} profitiert davon ebenfalls, da das Prinzip dasselbe ist.\\
Es können keine genauen Zahlen angegeben werden, welche Scores diese Methoden ohne Preprocessing-Schritte erreichen würden, da diese Schritte zwingend benötigt werden, um den Text in eine klassifizierbare Form zu bringen.
\subsection{Klassifizieren mittels Machine-Learning}
\subsubsection{Einfache Preprocessingschritte}
Das Anwenden von einfachen Preprocessingschritten hat bei allen drei Feature-Extraction Methoden Verbesserungen bewirkt.\\
Lediglich bei der TF-IDF Methode verschlechterten sich gewisse Algorithmen mit dem einfachen Preprocessing.
Die Verbesserungen beim TF-IDF sind jedoch markanter als die Verschlechterungen.\\
Das einfache Preprocessing hatte auf alle drei Varianten positive Einwirkungen und wird somit auch bei allen dreien weiter verwendet.
\begin{figure}[H]	
	\includegraphics[width=1\columnwidth,keepaspectratio]{img/easypre.png}
	\caption{Grafik der Auswertung für einfaches Preprocessing}
\end{figure}
\subsubsection{Fortschrittliche Preprocessingschritte}
Das fortschrittliche Preprocessing erzielt bei allen drei Feature-Extraction Methoden keine eindeutigen Resultate.\\
Bei der \glqq Bag of Words\grqq{} Methode erreicht die Konfiguration \glqq config8\grqq{} mit dem Preisdetektor, der Stoppwörterentfernung und der Stammformreduktion die grössten positiven Einwirkungen.
Sieben Algorithmen können mit dieser Konfiguration ihre F1-Scores verbessern.
Somit wird für \glqq Bag of Words\grqq{} die Konfiguration \glqq config8\grqq{} weiter benutzt.\\\\
Bei der binären \glqq Bag of Words\grqq{} Methode gibt es bei keiner Konfiguration irgendwelche flächendeckenden Verbesserungen.
Es gibt jeweils nur vereinzelte Algorithmen, welche ihre Scores verbessern können.
Da keine Konfiguration eindeutig als Verbesserung angesehen werden kann, wird bei dieser Variante keine fortschrittlichen Preprocessingschritte angewendet.\\\\
Bei der TF-IDF-Methode gibt es ebenfalls keine eindeutigen Verbesserungen bei irgendeiner Konfiguration.
Es können vereinzelte Algorithmen ihre Scores verbessern, aber es findet nie flächendeckend eine Verbesserung statt.
Bei der Konfiguration \glqq config5\grqq{} erzielt AdaBoost den höchsten Wert über alle Konfigurationen gesehen, aber die anderen Algorithmen werden nur leicht beeinflusst.
Um bei der weiteren Ermittlung von Optimierungen nicht nur auf ein Algorithmus zu setzen, wird bei TF-IDF kein fortschrittliches Preprocessing angewendet.
\\\\
Die Auswertung für das fortschrittliche Preprocessing kann im Anhang gefunden werden.
\section{Klassifizieren mittels Machine-Learning}
\subsection{Dimensionsreduktion der Features}
Die Verwendung der Dimensionsreduktion mittels LSA erzielt bei allen drei Feature-Extraction Methoden deutliche Verbesserungen.
Der Grossteil der Algorithmen kann mit Hilfe von LSA den F1-Score verbessern.
Lediglich bei der \glqq Bag of Words\grqq{} Methode ist die Summe über alle Score-Verbesserungen negativ, dies jedoch nur, weil der Multinomial-Bayes Klassifizierer mit LSA einen stetigen F1-Score von null hat.\\
Die Dimensionsreduktion wird für alle drei Varianten weiter verwendet, da die positiven Auswirkungen sich deutlich in den F1-Scores widerspiegelt.
Einziger Wehrmutstropfen der Dimensionsreduktion ist, dass der Multinomial-Bayes Klassifizierer keine wirkliche Klassifizierung mehr durchführen kann.
Somit ist dieser Klassifizierer für den weiteren Verlauf nicht mehr verwendbar.
\begin{figure}[H]	
	\includegraphics[width=1\columnwidth,keepaspectratio]{img/dimred.png}
	\caption{Grafik der Auswertung für die Dimensionsreduktion mittels LSA}
\end{figure}
\subsection{Klassenverteilung}
Die Angabe der Klassenverteilung konnte nicht bei allen Algorithmen als Parameter angegeben werden.
Somit werden alle Bayes-Algorithmen, KNearestNeighbor und Nearest-Centroid in dieser Auswertung nicht beachtet.\\
Bei beiden \glqq Bag of Words\grqq{} Methoden erzielt die Angabe der Klassenverteilung eine durchschnittliche Verbesserung der F1-Scores.
Einzelne Algorithmen reagieren negativ auf die Angabe, aber die positiven Auswirkungen übertreffen die negativen Auswirkungen.
Somit wird bei beiden \glqq Bag of Words\grqq{} Methoden die Angabe der Klassenverteilung miteinbezogen.\\\\
Bei der TF-IDF-Methode werden drei Algorithmen negativ und vier Algorithmen positiv beeinflusst.
Lediglich beim PassiveAgressiveClassifier gibt es eine markante Verschlechterung des F1-Scores.
Bei den anderen Algorithmen halten sich die negativen Auswirkungen im Rahmen.
Den F1-Score des PassiveAgressiveClassifier ausgenommen, sind die positiven Auswirkungen grösser als die negativen Auswirkungen.
Somit wird für TF-IDF ebenfalls die Angabe der Klassenverteilung für die nächsten Schritte beibehalten.
\begin{figure}[H]	
	\includegraphics[width=1\columnwidth,keepaspectratio]{img/classweight.png}
	\caption{Grafik der Auswertung für die Klassenverteilung}
\end{figure}
\subsection{N-Gramme}
Sowohl \glqq Bag of Words\grqq{} als auch TF-IDF bieten in ihrer Scikit-Learn Implementierung die Möglichkeit N-Gramme zu verwenden.
Für dieses Experiment wurden Unigramme, Bigramme und Trigramme ausgetestet.\\
Bei beiden \glqq Bag of Words\grqq{} Methoden erzielt die Verwendung von Unigrammen die besten Ergebnisse.
Bi- und Trigramme können bei einzelnen Algorithmen ebenfalls eine Verbesserunge verzeichnen, aber im Vergleich zu Unigrammen fallen diese flächendeckend kleiner aus.
Somit werden bei beiden \glqq Bag of Words\grqq{} Varianten nur Unigramme verwendet.
Ebenfalls benötigt die Extrahierung der Features bei Bi- und Trigrammen circa doppelt so lange, was eine enorme Performanceeinbusse ist.\\\\
Bei der TF-IDF-Methode erzielen Bi- und Trigramme die exakt gleichen Werte.
Uni- und Bigramme erzielen im Durchschnitt ungefähr den gleichen Wert.
Bei der Verwendung von Unigrammen erzielt der Algorithmus Bernoulli-Bayes einen F1-Score von null.
Um nicht einen weiteren Algorithmus für die weiteren Schritte zu verlieren, wird deshalb die Verwendung von Bigrammen bei der TF-IDF-Methode verwendet.
\begin{figure}[H]	
	\includegraphics[width=1\columnwidth,keepaspectratio]{img/ngram.png}
	\caption{Grafik der N-Gramme Auswertung}
\end{figure}
\subsection{Anzahl extrahierter Features}
Für alle drei Feature-Extraction Methoden wurde ein Liniendiagramm für F1-Score, Precision und Recall erstellt.
Aus diesen Grafiken kann entnommen werden, bei welcher Feature-Anzahl welcher Algorithmus den besten Score erzielt.
Für die weitere Auswertung sind die Grafiken für F1-Score und Precision massgebend und werden weiter analysiert.
\subsubsection{Bag of Words}
Bei \glqq Bag of Words\grqq{} erzielt der Algorithmus AdaBoost bei 100 Features mit Abstand den besten F1-Score und erreicht fast die 0.8 Marke.
Ebenfalls kann der LinearSVC Algorithmus einen guten F1-Score erzielen und teilt sich mit AdaBoost die besten Werte.
Ersichtlich ist ebenfalls, dass die meisten Algorithmen sich zwischen 0.6 und 0.7 bewegen und das vereinzelte Klassifizierer mit ihren Werten auf und ab springen.\\
\begin{figure}[H]	
	\includegraphics[width=0.8\columnwidth,keepaspectratio]{img/bow-f1.png}
	\caption{Grafik des F1-Score-Verlaufs bei Bag of Words}
\end{figure}
Bei der Precision ist der RandomForest Algorithmus der mit Abstand beste Klassifizierer.
Vereinzelt erreicht RandomForest eine Precision von 1.0 und ist stetig über 0.8.\\
Der Bernoulli-Bayes Algorithmus erreicht kurzzeitig ebenfalls eine Precision von eins, jedoch ist sein F1-Score jeweils unter 0.1.
Deswegen wir Bernoulli-Bayes nicht als ernsthafte Wahl angesehen und nicht beachtet.
\begin{figure}[H]	
	\includegraphics[width=0.8\columnwidth,keepaspectratio]{img/bow-pre.png}
	\caption{Grafik des Precision-Verlaufs bei Bag of Words}
\end{figure}
\subsubsection{Binäres Bag of Words}
Bei der Methode \glqq binäres Bag of Words\grqq{} können mehrere Algorithmen einen F1-Score nahe der 0.8 Grenze verbuchen.
Der beste Algorithmus ist Perceptron, welcher mit 325 Features einen F1-Score von 0.8 verzeichnet.\\
Der SGDClassifier erreicht ebenfalls einen F1-Score von 0.8, jedoch mit einer höheren Anzahl von Features.
Dies bedeutet das SGDClasifier potenziell länger für die Feature-Extraction benötigt und somit ist Perceptron der favorisierende Algorithmus bei dieser Variante.\\
\begin{figure}[H]	
	\includegraphics[width=0.8\columnwidth,keepaspectratio]{img/bow-bin-f1.png}
	\caption{Grafik des F1-Score-Verlaufs bei binärem Bag of Words}
\end{figure}
Bei der Precision ist der RandomForest Algorithmus ebenfalls die beste Wahl.
Er erreicht wieder die höchsten Prcecision-Werte bei einer vernünftigen Anzahl Features.\\
Ebenfalls springt der Bernoulli-Bayes wieder zwischen 1.0 und 0.5 umher.
Bei dieser Variante wird der Bernoulli-Bayes als keine Alternative gegenüber dem RandomForest in Betracht gezogen.
\begin{figure}[H]	
	\includegraphics[width=0.8\columnwidth,keepaspectratio]{img/bow-bin-pre.png}
	\caption{Grafik des Precision-Verlaufs bei binärem Bag of Words}
\end{figure}
\subsubsection{TF-IDF}
Bei der Verwendung von TF-IDF sind mehrere Algorithmen mit ihren F1-Scores im Bereich 0.7 bis 0.8.
Der SGDClassifier Algorithmus kann als einziger die Grenze von 0.8 durchbrechen und ist somit der Algorithmus mit dem besten F1-Score.\\
\begin{figure}[H]	
	\includegraphics[width=0.8\columnwidth,keepaspectratio]{img/tfidf-f1.png}
	\caption{Grafik des F1-Score-Verlaufs bei TF-IDF}
\end{figure}
Bei der Precision ist der RandomForest Algorithmus ebenfalls wieder der Spitzenreiter.
Er erzielt bei fast jeder Anzahl von Features die beste Precision und kann bei 325 Features sein Maximum erreichen.
\begin{figure}[H]	
	\includegraphics[width=0.8\columnwidth,keepaspectratio]{img/tfidf-pre.png}
	\caption{Grafik des Precision-Verlaufs bei TF-IDF}
\end{figure}
\subsection{Hyperparametertuning}
Die sechs Modelle, welche aus dem vorherigen Experiment, als die besten entnommen wurden, werden mittels Kreuzvalidierung auf optimale Hyperparameter durchsucht.
\subsubsection{Modelle mit bestem F1-Score}
Die drei Modelle AdaBoost, SGDClassifier und Perceptron können die besten F1-Scores erzielen.\\
\begin{table}[H]
	\caption{Auwertung Hyperparametertuning für AdaBoost mit binärem Bag of Words}
	\centering
	\begin{tabular}{|l|l|l|l|}
		\hline
		 & F1-Score & Precision & Recall\\
		\hline
		Vor Hyperparametertuning & 0.778 & 0.837 & 0.727 \\
		Nach Hyperparametertuning & 0.788 & 0.817 & 0.761 \\
		\hline
	\end{tabular}
\end{table}
\begin{table}[H]
	\caption{Auwertung Hyperparametertuning für Perceptron mit Bag of Words}
	\centering
	\begin{tabular}{|l|l|l|l|}
		\hline
		& F1-Score & Precision & Recall\\
		\hline
		Vor Hyperparametertuning & 0.8 & 0.872 & 0.739 \\
		Nach Hyperparametertuning & 0.579 & 0.426 & 0.903 \\
		\hline
	\end{tabular}
\end{table}
\begin{table}[H]
	\caption{Auwertung Hyperparametertuning für SGDClassifier mit TF-IDF}
	\centering
	\begin{tabular}{|l|l|l|l|}
		\hline
		& F1-Score & Precision & Recall\\
		\hline
		Vor Hyperparametertuning & 0.805 & 0.768 & 0.847 \\
		Nach Hyperparametertuning & 0.762 & 0.682 & 0.864 \\
		\hline
	\end{tabular}
\end{table}
Auffällig ist, dass AdaBoost als einziges Modell bessere Hyperparameter mittels Hyperparemtertuning finden konnte.\\
AdaBoost kann seinen Recall verbessern und gleichzeitig verschlechtert sich seine Precision.
Da die Recallsteigerung grösser als der Precisionabfall ist, wird der F1-Score nach oben korrigiert.\\
Die anderen beiden Modelle erzielen mit den neuen Parametern schlechtere Werte.
Beide können den Recall verbessern, jedoch müssen sie massive Gefälle bei der Precision einbüssen.
Da die Precision viel stärker abgenommen, als der Recall zugenommen hat, wird der F1-Score schlechter.\\
Perceptron hat im Vergleich zum SGDClassifier einen leicht tieferen F1-Score, jedoch ist seine Precision deutlich höher.
Da für den schlussendlichen \glqq Use-Case\grqq{} Precision wichtig ist, wird das Perceptron-Modell für weitere Auswertungen verwendet.
\subsubsection{Modelle mit bester Precision}
Das Modell RandomForest kann für alle drei Feature-Extraction Methoden jeweils den besten Precision-Score erzielen.\\
\begin{table}[H]
	\caption{Auwertung Hyperparametertuning für RandomForest mit binärem Bag of Words}
	\centering
	\begin{tabular}{|l|l|l|l|}
		\hline
		& F1-Score & Precision & Recall\\
		\hline
		Vor Hyperparametertuning & 0.636 & 1.0 & 0.466 \\
		Nach Hyperparametertuning & 0.736 & 0.8 & 0.682 \\
		\hline
	\end{tabular}
\end{table}
\begin{table}[H]
	\caption{Auwertung Hyperparametertuning für RandomForest mit Bag of Words}
	\centering
	\begin{tabular}{|l|l|l|l|}
		\hline
		& F1-Score & Precision & Recall\\
		\hline
		Vor Hyperparametertuning & 0.636 & 1.0 & 0.466 \\
		Nach Hyperparametertuning & 0.768 & 0.829 & 0.716 \\
		\hline
	\end{tabular}
\end{table}
\begin{table}[H]
	\caption{Auwertung Hyperparametertuning für RandomForest mit TF-IDF}
	\centering
	\begin{tabular}{|l|l|l|l|}
		\hline
		& F1-Score & Precision & Recall\\
		\hline
		Vor Hyperparametertuning & 0.747 & 0.956 & 0.614 \\
		Nach Hyperparametertuning & 0.794 & 0.823 & 0.767 \\
		\hline
	\end{tabular}
\end{table}
Alle drei Varianten können ihren F1-Score verbessern.
Die Verbesserung erfolgt nur im Bereich Recall, welcher initial bei allen drei Modellen relativ tief war.
Zusätzlich sinken bei allen drei Modellen die Precision-Scores.
Bei beiden Varianten mit Bag of Words, waren alle drei initialen Scores identisch und die Precision maximal.\\
Da in diesem Abschnitt auf die beste Precision geachtet wird, ist das Modell mit binärem Bag of Words und mit den Standardparametern die beste Variante.
\newpage

\chapter{Komponenten}
\section{Webcrawler}
\subsection{Beschreibung der Technologie}
\subsubsection{Apache Storm}
StormCrawler basiert auf Apache Storm, einem opensource Framework zur verteilten Stream-Verarbeitung.
Die Architektur von Apache Storm basiert auf den folgenden Komponenten:
\begin{itemize}
	\item Spout: Komponente zum Einlesen von Daten-Streams
	\item Bolt: Komponente zum Verarbeiten von Daten-Streams
	\item Tupel: Datensatz, welcher von Bolt zu Bolt weitergegeben wird
	\item Topologie: Ein Netz bestehend aus Spouts und Bolts
\end{itemize}
Daten können durch einen Spout aus verschiedenen Quellen (z.B. Datei, Datenbank, Array) eingelesen werden.
Diese Daten werden einem Tupel mitgegeben und an den nächsten Bolt in der Topologie weitergegeben.
\subsubsection{StormCrawler SDK}
StormCrawler ist ein opensource Software Development Kit, welches auf Apache Storm aufbaut und in Java entwickelt wurde.
Er dient als Baukasten, um einen Webcrawler aufzubauen, der skalierbar, stabil und sehr effizient ist.
Er beinhaltet verschiedene Spouts und Bolts, die explizit zum Crawlen von Websites vorgefertigt wurden.
Zudem berücksichtigt er die Regeln des Webcrawling, also Meta-Tags oder Robot.txt Dateien, welche deklarieren, ob eine Website gecrawlt werden darf.
StromCrawler ist standardmässig so eingerichtet, dass nur Webpages desselben Hosts gecrawlt werden.
\subsubsection{Docker}
\paragraph{Docker Einführung}
Docker besitzt verschiedene Bedeutungen. Docker kann sich auf den Namen der Firma Docker Inc."fussnote" beziehen oder auf die eigentliche Software.
In diesem Kontext wird Docker als die Software zum Erstellen von Linux-Containern verwendet.
Mithilfe von Docker können ressourcenarme, modulare virtuelle Maschinen erstellt werden, welche auch Container genannt werden.
Diese Container können dann einfach kopiert, ersetzt oder auf einem anderen System verwendet werden.
\footnote{\url{https://www.redhat.com/de/topics/containers/what-is-docker}}

\paragraph{Unterschiede zu Virtual Machines}
Im Gegensatz zu kompletten Virtual Machines (VM) besitzen Container einen deutlich kleineren Speicherfootprint.
VMs abstrahieren physische Hardware zu virtuellen Hardwarekomponenten. Auf diesen virtuellen Hardwarekomponenten können dann Betriebssysteme
und Software betrieben werden ähnlich wie auf der physischen Hardware.
Container abstrahieren jedoch auf Betriebssystemebene.
Dadurch bekommt man eine virtuelle Variante des physischen Host-Systems, auf dem man anschliessend Software und Abhängigkeiten betreiben kann.
Für die Containerisierung ist es notwendig, dass das Host-System eine Linux-Distribution ist.
\footnote{\url{https://wirtschaftslexikon.gabler.de/definition/hardware-virtualisierung-53364}}

\paragraph{Linux-Container}
Linux-Container verwenden den Linux-Kernel und seine Funktionen um Isolation der einzelnen Prozesse zu realisieren.
Durch die Isolation der Prozesse können diese unabhängig voneinander ausgeführt werden,
was die Auslastung des Host-Systems verbessert und ebenfalls die Sicherheit des Gesamtsystems erhöht.
Der Vorteil von Containern ist, dass alle Abhängigkeiten mit der ausführbaren Software in den Container gepackt werden können und somit jede
Umgebung, die den gleichen Container verwendet, auch den gleichen Softwarestand besitzt.
Dadurch wird die Software auf jeder Umgebung identisch ausgeführt.
Docker selbst baut auf der LXC-Technologie, eine Technologie zur Virtualisierung von Linux-Container auf Betriebssystemebene, auf,
erweiterte diese Technologie jedoch für das einfachere Erstellen, Warten und Versionieren von den erstellten Docker Containern.
\footnote{\url{https://www.redhat.com/de/topics/containers/what-is-docker}}

\paragraph{Docker}
Wie bereits erwähnt, wird für die Verwendung von Docker als Containerisierungs-Software eine Linux-Distribution als Hostsystem benötigt.
Für Windows- oder MAC-Systeme wird Software bereitgestellt, welche reduzierte Linux-VMs als Basis für die Verwendung von Docker installieren.
%\footnote{\url{https://docs.docker.com/docker-for-mac/docker-toolbox/#the-docker-for-mac-environment}}
%\footnote{\url{https://docs.docker.com/docker-for-windows/install/#what-to-know-before-you-install}}

\paragraph{Docker vorteile}
Die Vorteile von Docker wurden oben bereits kurz erwähnt, werden hier aber nochmals aufgelistet.
\begin{itemize}
	\item Docker zum Ausführen von containerisierter Software ist ressourcenärmer als die standardmässigen VMs
	\item Containerisierte Software wird auf jeder Umgebung identisch ausgeführt, dadurch kann das Ausliefern von Software immens vereinfacht werden
	\item Container können wahlweise dazugeschaltet werden, falls der Service überdurchschnittlich stark ausgelastet wird
	\item Mehrere Container können mit passender Software zentral orchestriert werden, wodurch das Warten von Micro-Services Architekturen vereinfacht wird
	\item Die Isolation der Container erhöht die Systemsicherheit, da die einzelnen Prozesse gekapselt werden
\end{itemize}
\paragraph{Dockerfiles}
Docker Container werden mit einem einzigen Konfigurationsfile beschrieben.
Dieses Konfigurationsfile, das sogenannte Dockerfile, definiert alle Notwendigkeiten, um den Docker Container zu realisieren.
Im Dockerfile werden Instruktionen für das Erstellen des Images definiert.
Der prinzipielle Aufbau eines Dockerfiles ist meist nach dem folgenden Schema aufgebaut.
\begin{lstlisting}
FROM ubuntu:15.04
COPY . /app
RUN make /app
CMD python /app/app.py
\end{lstlisting}
\begin{itemize}
	\item FROM  verwendet ein bereits erstelltes Docker Image als Basis für diese Image.
	\item COPY  kopiert Dateien, generell die geschriebene Software, vom Hostsystem in das Docker Image.
	\item RUN   erstellt die Software mit dem passenden Befehl, in diesem Fall wäre das der Befehl "make".
	\item CMD   definiert, welcher Befehl nachher im eigentlichen Container ausgeführt werden soll, generell wird damit Software gestartet.
\end{itemize}
Da Dockerfiles einfache Textdateien sind, können sie, wie jeder andere Sourcecode, mit Git, SVN oder anderen Versionsverwaltungen verwaltet werden.
Ausserdem bietet die Firma Docker Inc. eine Webseite namens "hub.docker.com" an, welche ähnlich wie "Github.com" die Versionierung von Dockerfiles vereinfacht.
Docker Images
Mit dem Dockerbefehl "docker build" wird aus dem Dockerfile ein Docker Image erstellt.
Das Docker Image wird wiederum verwendet, um einen ausführbaren Docker Container zu erstellen.
Mit dem Dockerbefehl "docker run" und der Angabe des gewünschten Images kann nun der Docker Container erstellt und gestartet werden.
Grundsätzlich kann das Docker Image mit einer Klasse im Objekt-Orientierten-Programmieren verglichen werden und der Docker Container wäre eine laufende Instanz der entsprechenden Klasse. 
\footnote{\url{https://stackoverflow.com/questions/3175105/inserting-code-in-this-latex-document-with-indentation}}
\footnote{\url{https://docs.docker.com/develop/develop-images/dockerfile_best-practices/}}

\paragraph{Docker-Compose}
Mit Docker Containern können auch Micro-Services Architekturen realisiert werden, wodurch jeder Container einen eigenen Service darstellt.
Das manuelle Starten jedes einzelnen Containers erweist sich jedoch als ineffizient, deswegen wird die Software "Docker-Compose" ebenfalls von Docker Inc. bereitgestellt.
Mithilfe von Docker-Compose können mehrere Docker Container verknüpft und gleichzeitig gestartet werden.
Die Abhängigkeiten zwischen den Containern wird ebenfalls mit einem Konfigurationsfile, dem Composefile, definiert.
\begin{lstlisting}
version: '3'
services:
web:
build: .
ports:
- "8080:5000"
redis:
image: "redis:alpine"
\end{lstlisting}
\begin{itemize}
	\item version definiert, welche Version von docker-compose verwendet werden soll.
	\item web ist der erste Container, welcher erstellt werden soll.
	\begin{itemize}
		\item build definiert, wo sich das Dockerfile befindet, mit welchem ein Image erzeugt werden soll.
		\item ports definieren, welche Ports für den Service verwendet werden sollen. (Host hat Port 8080, Container intern hat Port 5000) 
	\end{itemize}
	\item redis ist der zweite Container, welcher erstellt werden soll.
	\begin{itemize}
		\item image definiert, welches bereits existierende Image für den Container verwendet werden soll.
	\end{itemize}
\end{itemize}
Wahlweise können noch Dateiordner vom Hostsystem in einzelne Container angehängt werden, damit die Container Daten dort abspeichern oder abrufen können.
Dies ist insofern wichtig, da sobald die Container gestoppt werden, all ihre internen Daten verloren gehen.
Die Containerinhalte existieren nur so lange, wie auch die Container existieren.
Mit dem Docker-Compose-Befehl "docker-compose up" wird das im gleichen Pfad existierende Composefile gestartet und alle Container erstellt.
Docker-Compose erstellt für alle Container in einer Gruppierung einzigartige ID-Namen, damit alle Container innerhalb der Gruppierung miteinander kommunizieren können.
\footnote{\url{https://github.com/DigitalPebble/storm-crawler/}}
\subsection{Konfiguration}
Die Grundlage der Konfiguration ist die Standardtopologie des Stromcrawlers auf Github.
\footnote{\url{https://docs.docker.com/develop/develop-images/dockerfile_best-practices/}}
Es wurde ein neuer Bolt erstellt, welcher für das Schreiben des Outputs zuständig ist.
Dieser schreibt für jede Webpage eine JSON Datei mit den folgenden Informationen:
\begin{itemize}
	\item \glqq date \grqq  - Zeitpunkt, zu welchem die Webpage aufgerufen wurde
	\item \glqq text \grqq  - Vom Webcrawler extrahierter Text, welcher die Webpage beinhaltet
	\item \glqq encoding \grqq  - Das von der Webpage verwendete Encoding
	\item \glqq title \grqq  - Inhalt des gleichnamigen HTML-Metatags
	\item \glqq url \grqq  - URL der Webpage
	\item \glqq content \grqq  - Der statische HTML-Inhalt der Webpage	
\end{itemize}
Der Dateiname wird aus der URL der Webpage generiert. 
Sonderzeichen, die in Dateinamen nicht erlaubt sind, werden entfernt und falls die URL länger ist, als die erlaubte Dateinamenslänge, wird dieser abgeschnitten und mit einem zufälligen vierstelligen Suffix erweitert.
Zudem kommt darin eine Spracherkennung zum Einsatz, welche anhand des Textes einer Webpage detektiert, ob sie mehrheitlich in deutsch geschrieben wurde.
Als deutsch detektierte Webpages werden in einen separaten Output-Ordner gespeichert, damit diese nachfolgend von Hand gelabelt werden können.
Trotzdem ist es wichtig, dass alle aufgerufenen Webpages gespeichert werden, damit im Anschluss des Crawlens eine Aussage gemacht werden kann, wie viele Einträge des Seeds effektiv gecrawlt wurden.

\section{Klassifizierung}
\subsection{Verwendete Technologien}
\subsubsection{Luigi}
Luigi ist ein Pipelining-Tool, welches von Spotify entwickelt und später als Open-Source Projekt veröffentlicht worden ist.
Pipelining dient dazu da, mehrere Tasks miteinander zu verknüpfen, das Ausführen zu Automatisieren und somit eine grössere Aufgabe zu verrichten.
Pipelining wird meist im Kontext von Big Data oder Machine-Learning angewendet, wo sich viele einzelne Tasks mit grossen Datenmenge beschäftigen.
Luigi wird selbst von Spotify in der Produktion verwendet und bietet unteranderem folgende Features:
\begin{itemize}
	\item Einzelne Tasks idempotent ausführen
	\item Teilschritte oder gesamte Pipeline kann über Konsolenausgabe oder Webinterface überwacht werden
	\item Fehlerfälle können protokolliert werden und dementsprechend reagiert werden
	\item Abhängigkeiten von Tasks werden selbstständig von Luigi gelöst
	\item Luigi ist komplett in Python aufgebaut und kann auch mit Python konfiguriert werden
\end{itemize}
Die einzelnen Tasks werden mit Python-Klassen realisiert.
Luigi ist so aufgebaut, dass jeder Task ein Input-File liest, von dem die Ausgabe des vorherigen Tasks eingelesen wird und ein Output-File schreibt, welches als Input-File des anschliessenden Tasks benutzt wird.
Dadurch kann Luigi die Zustände der einzelnen Tasks überwachen und bei einem Fehlerfall die Pipeline dort wieder starten, wo sie sich aufgehängt hat.
Damit Luigi die Abhängigkeiten und das Überwachen bewerkstelligen kann, benötigt jede Klasse in der Pipeline folgende 3 Funktionen:
\begin{itemize}
	\item Funktion requires(), Angabe auf welche Tasks Abhängigkeiten bestehen
	\item Funktion run(), Bereich, wo die eigentliche Logik des Tasks ist
	\item Funktion output(), wohin die Ausgabe geschrieben wird
\end{itemize}
%######### foto von https://luigi.readthedocs.io/en/stable/tasks.html
In dieser Arbeit wurde Luigi verwendet, um die Klassifizierung zu verknüpfen.
--------------- Produktions- und Entwicklungspipeline erläutern
Es wurde zwei verschiedene Arten von Pipelines erstellt.
Die Entwicklungspipeline diente der Entwicklung der Regeln für die Klassifizierung.
Die Klassifizierung beinhaltet das Lesen der gecrawlten Daten, das Preprocessing, das Klassifizieren und das Evaluieren der Klassifizierung.
Die daraus folgende Pipeline beinhaltet folgende Tasks:
\begin{itemize}
	\item Importer, import die gecrawlten Websites und speichert sie in einer CSV-Datei
	\item Preprocessor, wendet übliche Preprocessing-Schritte an den Daten an
	\item RuleBasedClassifier, regelbasierte Klassifizierung der Daten 
	\item MLClassifier, Klassifizierung der Daten mittels Machine-Learning
	\item Evaluator, Auswertung der Klassifizierung von Daten (Nur im Entwicklunsmodus)
\end{itemize}
\subsubsection{Scikit Learn}
\subsection{Preprocessing}
Das Preprocessing wurde entwickelt, um den Inhalt des Dokuments auf eine standardisierte Form zu bringen.
Die folgenden Methoden können sowohl auf den Text eines Dokuments sowie auch auf den Titel angewendet werden.
Diese werden sequenziell in der unten aufgeführten Reihenfolge angewandt und können mittels Konfiguration sowohl für den Text als auch Titel eines Dokuments ein- oder ausgeschaltet werden.
\subsubsection{Gross-/Kleinschreibung}
Alle Buchstaben, welche grossgeschrieben sind, werden durch Kleinbuchstaben ersetzt.
\subsubsection{Umlaute ersetzen}
Umlaute werden durch ihre verwandten Selbstlaute ersetzt, genauer:
\begin{itemize}
	\item ä $\rightarrow$ a
	\item ö $\rightarrow$ o
	\item ü $\rightarrow$ u
\end{itemize} 
\subsubsection{Preisdetektor}
Da Menüs häufig in Verbindung mit Preisen vorkommen und in weiteren Preprocessing-Schritten Zahlen und Sonderzeichen entfernt werden, ist es von Vorteil, diese Informationen nicht zu verlieren.
Daher erkennt diese Methode verschiedene Varianten von Preisen mittels Regulären Ausdrücken (Regex, Regular Expression) und ersetzt diese mit einem Schlüsselwort.
% Varianten genauer ausführen
Die folgenden Varianten von Preisen wird erkannt:
\begin{itemize}
	\item preisangabe + chf/fr/sfr
	\item preisangabe
	\item chf/fr/sfr + preisangabe
\end{itemize} 
Zudem wird zwischen unterschieden, wie viele Stellen der Preis hat.
Die nun aufgeführte Liste zeigt die verschiedenen Schlüsselwörter:
\begin{itemize}
	\item Einstellig $\rightarrow$ onedigitprice
	\item Zweistellig $\rightarrow$ twodigitprice
	\item Dreistellig $\rightarrow$ threedigitprice
\end{itemize} 
Um Zeitangaben nicht als Preise zu erkennen, werden nur Rappenbeträge, welche 60 oder höher sind, erkannt.
\subsubsection{Sonderzeichen entfernen}
Alle Sonderzeichen, die nicht in der folgenden Auflistung vorkommen, werden durch einen Leerschlag ersetzt:
%Eventuell noch anpassen, da Grossbuchstaben nicht mehr relevant sind (EVTL. é usw auch noch ersetzen?)
\begin{itemize}
	\item éàèÉÀÈäöüÄÖÜa-zA-Z
\end{itemize} 
\subsubsection{Einzelne Zeichen entfernen}
Jedes einzelne Zeichen, also solche, die sowohl vorne als auch hinten an einen Leerschlag angrenzen, werden entfernt.
\subsubsection{Multiple Leerschläge entfernen}
Da durch die vorhergehenden Schritte oft multiple Leerschläge anfallen, werden diese auf einen Leerschlag reduziert.
\subsubsection{Stammformreduktion}
Dieses Verfahren führt verschiedene morphologische Varianten eines Wortes auf ihren gemeinsamen Stamm zurück.
Dafür wird der Stemmer \glqq Cistem\footnote{\url{https://github.com/LeonieWeissweiler/CISTEM}} \grqq verwendet, da für die deutsche Sprache nur wenig Alternativen vorhanden sind. 
\subsubsection{Getränkedetektor}
% Referenz Cistem
Eine Liste mit Einträgen verschiedenster Getränke bildet die Grundlage dieser Methode.
Wenn im Text ein Getränk dieser Liste vorhanden ist, wird es durch das Schlüsselwort \glqq beverageentity\grqq ersetzt.
Damit soll erreicht werden, dass ein einheitliches Merkmal geschaffen wird.
\subsubsection{Stoppwörter entfernen}
Bei Stoppwörter handelt es sich um Wörter, welche keine Relevanz für den Inhalt eines Texts haben, aber oft vorkommen.
Eine Stoppwortliste führt 1720 solcher Wörter in deutsch auf. Sie ist aus mehreren Quellen zusammengesetzt worden.
% Quellen Stoppwortliste aufführen?
Wenn eines dieser Wörter im Text vorkommt, wird es entfernt.
\subsection{Regelbasierte Klassifizierung}
Verschiedene Algorithmen kommen beim regelbasierten Klassifizieren zum Einsatz, welche nun genauer erläutert werden.
\subsubsection{Menü im Titel}
Dieser simple Algorithmus kontrolliert, ob das Wort "menu" im Metatag \glqq Title \grqq vorkommt.
Falls ja, wird die Webpage als Menüseite klassiert.
\subsubsection{Preisdetektor}
Dieser Algorithmus funktioniert dankt der Preprocessing-Methode, welche den Preis erkennt.
Sofern ein zweistelliger Preis erkannt wurde, wird die Webpage als Menüseite klassifiziert, dies unter der Annahme, dass Menüpreise oft zweistellig sind, im Gegensatz zu Getränkepreisen welche häufig einstellig sind und Hotelpreisen, die meist dreistellig sind.
Die Anzahl vorhandener Preise kann über die Konfiguration angegeben werden.
\subsubsection{Kombination aus Menü im Titel und Preisdetektor}
Diese Kombination führt eine sequenzielle Klassifikation aus.
Im ersten Schritt wird der Mechanismus des Algorithmus \glqq Menü im Titel \grqq verwendet.
Falls durch diesen keine positive Klassifikation zustande kommt, wird durch den Preisdetektor nochmals neu klassifiziert.
\subsubsection{Listing}
Dieser Algorithmus basiert sowohl auf dem Black- als auch auf dem Whitelisting Ansatz.
Dazu wird eine Blacklist und eine Whitelist verwendet, welche von Hand erstellt wurden und Einträge enthalten, welche für die jeweiligen Kategorien typisch sind.
Falls ein Wort einer Liste im Text einer Webpage vorkommt, wird ein Zähler hochgezählt.
Zum Schluss findet eine sequenzielle Klassifizierung statt, das heisst: 
Wenn der Zähler der Blacklist einen konfigurierbaren Schwellwert überschreitet, wird die Webpage als negativ klassifiziert. 
Wenn der Zähler der Whitelist einen konfigurierbaren Schwellwert überschreitet, wird die Webpage als Menüseite klassifiziert.
Falls keiner der beiden Zähler den Schwellwert überschreitet, wird die Webpage ebenfalls als negativ klassifiziert. 
\subsubsection{Bag of Words}
Dieser Algorithmus basiert auf dem statistischen Ansatz namens \glqq Bag of Words\grqq.
Bag of Words zählt für jede Webpage, wie oft ein Wort darin vorkommt.
In dieser Anwendung wird jedoch nur erkannt, ob ein Wort vorkommt, oder nicht, die Anzahl spielt keine Rolle.
Anhand dieser Wörter wird eine dynamische Black- und Whitelist erstellt, Wörter die in beiden Listen vorkommen, werden entfernt.
Um diese Listen erstellen zu können, werden die Daten in ein Trainings- und Testset unterteilt.
Anhand des Trainingssets werden die Listen erstellt, welche danach zur Klassifikation des Testsets dienen.
Die Anzahl der Wörter dieser Listen ist konfigurierbar.
Die Klassifikation findet wieder mittels Zähler statt.
Für jedes Wort eines Dokuments, welches in der Liste der positiven Beispielen vorkommt, wird der Zähler hochgezählt, für jedes Wort aus der Liste der negativen Beispielen wird er heruntergezählt.
Für die Klassifizierung wird der Zähler mit einem konfigurierbaren Schwellwert verglichen, Falls der Zähler grösser ist als der Schwellwert, wird das Dokument als positiv klassifiziert, ansonsten negativ.
\subsection{Feature-Extraction für Machine-Learning}
\subsection{Klassifizierung mittels Machine Learning}
\subsubsection{Konfiguration}
\subsubsection{}
\subsection{Evaluierung}
Da es sich in diesem Experiment um eine binäre Klassifizierung handelt, werden Metriken wie F1-Score, Precision und Recall verwendet.
Dazu müssen zuerst die folgenden Klassifizierungskategorien erläutert werden:
\begin{itemize}
	\item Richtig Positiv: Die Webpage wurde als Menüseite klassifiziert und ist eine Menüseite
	\item Falsch Positiv: Die Webpage wurde als Menüseite klassifiziert, ist aber keine Menüseite
	\item Falsch Negativ: Die Webpage wurde nicht als Menüseite klassifiziert, ist aber eine Menüseite
	\item Richtig Negativ: Die Webpage wurde nicht als Menüseite klassifiziert und ist keine Menüseite
\end{itemize}
Jede klassifizierte Webpage lässt sich in eine dieser Kategorie einordnen.
Anhand dieser Kategorien lässt sich für den gesamten Datensatz die oben genannten Metriken berechnen:\\
\[Precision=\frac{\text{Richtig Positiv}}{\text{Richtig Positiv} + \text{Falsch Positiv}}\]\\
Die Precision definiert das Verhältnis der korrekt positiv klassifizierten Menüseiten zu allen als positiv gelabelten Menüseiten.\\
\[Recall=\frac{\text{Richtig Positiv}}{\text{Richtig Positiv} + \text{Falsch Negativ}}\]\\
Der Recall definiert das Verhältnis der korrekt positiv klassifizierten Proben zu allen tatsächlichen Menüseiten.\\
\[F_{1}=\frac{\text{Precision} \times \text{Recall}}{\text{Precision} + \text{Recall}}\]\\
Anhand F1-Score wir die Genauigkeit der Klassifizierung gemessen.
Für jede Methode und jede Konfiguration der Klassifizierung werden diese Scores berechnet, um eine Aussage machen zu können, wie gut sie sind.
\section{Webapplikation}
\subsection{Beschreibung der Technologie}
\subsection{Beschreibung der Architektur}
\subsection{Search Engine}
\subsubsection{Beschreibung der Technologie}
\subsubsection{Konfiguration}
\newpage

\chapter{Vorgehen}
\section{Webcrawler}
\subsection{Erarbeitung des Seeds}
Als Quelle für das Abrufen von Websites dient das Seed.
Das Ziel war es, dieses mit möglichst vielen URLs von Schweizer Restaurant-Websites zu füllen.
Dafür wurden zwei Ansätze verfolgt.
Ein Verein Schweizer Restaurants, namentlich Lunch-Check\footnote{\url{https://www.lunch-check.ch/}} wurde angefragt, ob sie eine Liste ihrer Mitglieder zur Verfügung stellen.
Zudem wurde der OpenStreetmap-API\footnote{\url{https://www.openstreetmap.ch/}} genutzt, um die darin enthaltenen Restaurant-URLs abzufragen.
Die Daten aus beiden Quellen wurden zusammengeführt und dienen nun als Seed für die Abfragen des Webcrawlers.
\subsection{Erarbeitung des Webcrawlers}
Als erstes fand eine Einarbeitung in die Thematik von Docker statt, um StormCrawler damit zu betreiben.
Apache Storm ist ebenfalls thematisiert worden, da StormCrawler darauf basiert.
Danach folgte die Einarbeitung in StormCrawler.
Zu Beginn war das Ziel, die Standardtopologie zu verstehen.
Sobald die Standardtopologie funktionierte, fanden Anpassungen statt, welche zur Erstellung der Rohdaten für den Gold Standard dienten.
Explizit ist die Erstellung des Bolts zu erwähnen, welcher für das Schreiben der Output-Dateien zuständig ist.
Dieser wurde zudem mit einer Sprachdetektion erweitert, welche erkennt, ob die Sprache der Webpage deutsch ist.
\subsection{Verwendete Technologien}
\subsubsection{StormCrawler}
Es fand keine Evaluation zur Findung eines geeigneten Webcrawlers statt.
Stormcrawler wird eingesetzt, da an der Hochschule für Technik und Wirtschaft Chur ein Team bereits mit diesem arbeitet und dadurch Know-how besitzt.
Dieses Team, speziell ein wissenschaftlicher Mitarbeiter, hat Support angeboten, weshalb die Entscheidung getroffen wurde, Stormcrawler zu verwenden.
\subsubsection{Docker}
Stormcrawler wurde nicht direkt installiert, sondern mittels Docker-Container aufgesetzt.
Dies aus dem Grund, dass dadurch eine Skalierung einfach möglich ist.
Zudem ist die Installation hinfällig, da auf Docker Hub\footnote{\url{https://hub.docker.com/}} bereits fertige Images von Apache Storm und StormCrawler vorhanden sind.
\section{Gold Standard}
\subsection{Crawlen der Rohdaten}
Das Crawlen der Rohdaten war mit diversen Komplikationen verbunden.
Die Performance des Webcrawlers war zu Beginn nicht zufriedenstellend, es wurden lediglich ca. 30 Webpages pro Minute gespeichert.
Trotzdem wurde ein erster Rohdatensatz mit dieser Performance gecrawlt, auf dem das erste manuelle Labeling durchgeführt wurde.
Bei diesem Rohdatensatz wurden nur die als deutsch detektierten Webpages gespeichert, somit konnte die Abdeckung des Seeds nicht ausgewertet werden.
Der Crawler wurde nochmals angepasst, sodass alle Webpages gespeichert werden, dies ermöglicht eine genauere Analyse der Rohdaten.
Die Spracherkennung wurde zudem so angepasst, dass die Anzahl zu erkennender Sprachen eingeschränkt wurden und die Sprachdetektion nicht für jede Webpage neu gestartet wurde, was eine massive Performancesteigerung zur Folge hatte.
Diverse weitere Crawldurchläufe wurden gemacht und die gecrawlten Daten analysiert.
Diese Analysen haben ergeben, dass einzelne Websites aus dem Seed enorm viele Webpages beinhalten, zum Teil über 30'000.
Diese URLs dieser Websites wurden aus dem Seed entfernt, da sie keine typischen Restaurant-Webseiten repräsentieren.
Für die zweite Durchführung des manuellen Labelings wurden die Rohdaten mit dem angepassten Seed neu gecrawlt.
%- Auführen der Crawl-Analyse
\subsubsection{Erarbeitung des Entscheidungsrasters}
Das Entscheidungsraster ist die Grundlage des manuellen Labeling der Rohdaten.
Für dieses wurden die folgenden Entscheidungen getroffen:
\begin{itemize}
	\item Die Webpage muss auf Deutsch verfasst sein
	\item Der Anbieter muss entweder ein Restaurant, Take-Away oder Lieferdienst sein
	\item Der Text muss statisch im HTML vorhanden sein, da dynamisch gerenderte Informationen vom Webcrawler nicht gespeichert werden
	\item Es muss ein Menü, also eine Kombination aus mehreren Speisen oder eine einzelne Speise vorhanden sein
	\item Eine genauere Beschreibung oder der Preis muss vorhanden sein
	\item Getränkekarten werden explizit als negativ gelabelt
\end{itemize}
Bei der Klassifizierung wird zudem unterschieden, ob es sich um ein zeitlich begrenztes Angebot handelt, da diese Angebote zu einem späteren Zeitpunkt eventuell zusätzlich erkannt werden möchten.
\subsubsection{Manuelles Labeling der Daten}
In einer ersten Durchführung des manuellen Labelings wurden ca. 1500 Dateien klassifiziert, welche das Schlüsselwort \glqq Menu\grqq{} in der URL enthalten.
Die mit dieser Heuristik gefilterten Daten entsprechen jedoch nicht einer repräsentativen Teilmenge der gesamten Rohdaten.
Darum wurde in einer zweiten Durchführung zufällig Proben aus den Rohdaten ausgewählt und manuell gelabelt.
Dadurch konnte sichergestellt werden, dass das Verhältnis von Menüseiten zu den restlichen Webpages gleich bleibt.
\subsection{Verwendete Technologien}
\subsubsection{Labeling-Tool}
Um das manuelle Labeling effizienter zu gestalten, ist ein Tool\footnote{\url{https://github.com/s-santoro/testdata_tool}} erstellt worden, welches das Labeling vereinfacht.
Dieses Tool ruft eine Webpage der zufällig extrahierten Webpages aus den Rohdaten auf und zeigt sowohl den Text, als auch den HTML-Inhalt.
Der Anwender des Tools muss anhand dieser Informationen entscheiden, ob es sich um eine Webpage mit Menüinformationen handelt und ob diese zeitlich begrenzt sind.
Mittels Shortkeys findet die Klassifizierung statt.
Das Tool verschiebt die Datei in den entsprechenden Ordner und zeigt die Informationen der nächsten Webpage an.
\section{Klassifikation}
\subsection{Preprocessing}
Die Basisfunktionen des Preprocessings orientieren sich an einem Online-Artikel von Usman Malik\footnote{\url{https://stackabuse.com/text-classification-with-python-and-scikit-learn/}}.
Teile der darin beschriebenen Funktionen wurden angepasst und übernommen.
Weiter wurde erkannt, dass Preise mittels Regulären Ausdrücken erkannt werden müssen, damit diese Informationen nicht verloren gehen.
Darum ist eine Methode entwickelt worden, die diese Aufgabe erledigt.
Für die Funktion der Stammformreduktion ist nach einer entsprechenden Funktion gesucht worden, welche diese Aufgabe für die deutsche Sprache verrichtet.
Über die Dokumentation des Toolkits NLTK\footnote{\url{https://www.nltk.org/api/nltk.stem.html}} ist Cistem, ein Stemmer für die deutsche Sprache, gefunden und später als Preprocessing-Methode implementiert worden.
Zudem ist eine deutsche Stoppwortliste aus mehreren Quellen zusammengesetzt worden.
\subsection{Regelbasierte Klassifikation}
Die erstellten Regeln sind grösstenteils anhand Beobachtungen beim manuellen Labeln entstanden.
Dabei wurde erkannt, dass Preise sowie Speisenamen und Zutaten ein markantes Merkmal von Menüseiten sind.
Diese Informationen eignen sich für eine Whitelist.
Im Gegensatz gibt es viele Websites mit Webpages wie \glqq Imperessum\grqq{}, \glqq Anfahrt\grqq{} etc. welche typische Schlagwörter beinhalten, die in einer Blacklist aufgeführt werden können.
Durch die Regel \glqq Bag of Words\grqq{} wurde derselbe Ansatz wie beim Listing verfolgt, jedoch mit dem Ziel, durch die gleichnamige statistische Methode diese Listen dynamisch zu erzeugen.
Lediglich die Regel zur Erkennung des Schlagworts \glqq menü\grqq{} ist unter der Annahme entstanden, dass Webpages eine korrekte Angabe von Informationen im Metatag \glqq Title\grqq{} verwenden.
\subsubsection{Erstellen der Konfigurationen}
Die Konfiguration wurde erstellt, um einzelne Methoden des Preprocessings sowie die Art der Klassifizierung angeben zu können.
Wo möglich wurden Schwellwerte eingebaut, um die Klassifizierung eines Dokuments regulieren zu können.
Diese Schwellwerte wurden über mehrere Iterationen verändert, um möglichst hohe Scores zu erreichen.
Die Parameter dazu sind in \cref{cap:exp_class} ersichtlich.
\subsection{Machine-Learning Klassifikation}
\subsubsection{Einleitung}
Für die Klassifizierung mit Machine-Learning wurden mehrere Algorithmen jeweils mit den Metriken F1-Score, Recall und Precision ausgewertet.
Wegen des \glqq No free lunch\grqq{}-Theorems wurden insgesamt 14 unterschiedliche Algorithmen aus zwei unterschiedlichen Online-Artikeln\footnote{\url{https://scikit-learn.org/stable/auto_examples/text/plot_document_classification_20newsgroups.html}}\footnote{\url{https://medium.com/@robert.salgado/multiclass-text-classification-from-start-to-finish-f616a8642538}} zusammengetragen und stetig anhand ihrer Metriken verglichen.
Alle Algorithmen wurden aus Online-Artikeln entnommen, die ebenfalls eine textliche Klassifizierung verfolgten und diese Artikel dienten als Leitfaden für den Aufbau der ganzen Machine-Learning Pipelines.
Alle Algorithmen wurden jeweils mit ihrer Standardkonfiguration trainiert und anschliessend mit einem vorbehaltenen Validierungsset validiert, um einen möglichst unverzerrten Vergleich zu erzielen.
Schrittweise wurden für alle 14 Algorithmen folgende Pipelinekomponenten angepasst und versucht, einen optimalen Wert zu ermitteln.
Dabei wurde die Reihenfolge der Auflistung bei der schrittweisen Optimierung beibehalten.
Die Erkenntnisse aus den Optimierungen flossen immer beim nächsten Optimierungsschritt mit ein.
\begin{enumerate}
	\item Dimensionsreduktion der Features 
	\item Angabe von Klassenverteilung
	\item Anwendung von N-Gramme (N={1,2,3})
	\item Anwendung von einfachen Preprocessingschritten
	\item Anwendung von fortgeschrittenen Preprocessingschritten
	\item Anzahl extrahierter Features
\end{enumerate}
Ebenfalls wurde für die Feature-Extraction die Methoden Bag-of-Words mit Worthäufigkeit, TF-IDF und Bag-of-Words ohne Worthäufigkeit verwendet.
Die Ermittlung der optimalen Pipelinekomponenten erfolgte somit unabhängig für drei unterschiedliche Feature-Extraction Methoden und jeweils für 14 Algorithmen mit deren Standardkonfigurationen.
\subsubsection{Dimensionsreduktion der Features}
Bei der Dimensionsreduktion der Features wurde mittels LSA (Latent Sentiment Analysis)\footnote{\url{https://scikit-learn.org/stable/modules/decomposition.html}} versucht, aus einer Vielzahl von Features nur die Aussagekräftigsten zu ermitteln.
Dies wird erreicht, indem SLA Beziehungen von Wörtern mit ähnlicher Bedeutung erkennt.
Zuerst wurde über die API der drei Feature-Extraction Methoden die Anzahl Features direkt auf 100 Stück reduziert und ohne Verwendung von LSA an die Algorithmen weitergegeben.
Diese wurden dann trainiert, validiert und die Werte aufgenommen.
Anschliessend wurde für alle drei Feature-Extraction Methoden keine Begrenzung der Anzahl Features vorgenommen und erst mit LSA beziehungsweise der direkten Scikit-Learn Implementierung "TruncatedSVD" die Feature-Reduktion auf 100 Stück durchgeführt.
Danach wurden die Algorithmen mit den Features trainiert und validiert.
Die aus beiden Aushebungen ermittelten Werte wurde dann miteinander verglichen.
\subsubsection{Klassenverteilung}
Da die Daten aus dem Goldstandard eine sehr ungleiche Verteilung von positiven und negativen Samples aufweist, wurde die Klassengewichtung anhand dieser Verteilung berechnet.
Die Gewichtung, auf ein positives Sample folgen 10.32 negative Samples, wurde den Algorithmen fürs Trainieren mitgegeben.
Mittels der Klassenverteilung können die Algorithmen die unterrepräsentierten Klassen, in unserem Falle die positive Klasse, stärker gewichten und den Fokus darauf setzen.
Einerseits wurden die Algorithmen ohne Klassengewichtung trainiert und validiert und andererseits mit der Angabe der Klassengewichtung.
Zu erwähnen ist, dass die Gewichtung nicht allen Algorithmen zugewiesen werden konnte, da manche Algorithmen während dem Training solche Informationen nicht verwerten können.
\subsubsection{N-Gramme}
Die verwendeten Feature-Extraction Methoden bieten die Möglichkeit N-Gramme als Features zu extrahieren.
Aufgrund von verschiedenen Visualisierungen der Textdaten wurde erkannt, dass es Sinn macht, die Verwendung von Bi- und Trigrammen zu prüfen.
Die Scikit-Learn Feature-Extraction Methoden bieten einfache Schnittstellen für die Extrahierung von N-Grammen.
Die Validierung von Uni-, Bi- und Trigrammen wurde für alle drei Feature-Extraction Methoden durchgeführt, die 14 Algorithmen mit den neuen Features trainiert und wieder validiert.
\subsubsection{Einfaches Preprocessing}
Um die Textdaten auf eine vereinheitlichte Form zu bringen, wurden einfache Preprocessingschritte durchgeführt.
Als einfache Preprocessingschritte gelten im Kontext vom Machine-Learning Teil der Bachelorarbeit folgende Funktionen:
\begin{itemize}
\item Den gesamten Text auf Kleinbuchstaben umwandeln
\item Die Umlaute ersetzen
\item Sonderzeichen entfernen
\item Alleinstehende Zeichen entfernen
\item Multiple Leerzeichen entfernen
\end{itemize}
Diese Funktionen können im Unterkapitel \ref{preprocessing} nachgeschlagen werden.
Es wurde lediglich analysiert, was für Auswirkungen gar kein Preprocessing im Vergleich zu einfachem Preprocessing besitzt.
\subsubsection{Fortgeschrittenes Preprocessing}
Fortschrittliche Preprocessingschritte haben die Aufgabe den Text weiter zu modifizieren und die Extrahierung von zusätzlichen Informationen zu vereinfachen.
Im Kontext vom Machine-Learning Teil zählen folgende Preprocessingschritte zu den fortschrittlichen Funktionen:
\begin{itemize}
	\item Das Markieren von Preisen mit entsprechenden Tags
	\item Die Stammformreduktion
	\item Das Entfernen von Stopwörtern
	\item Das Markieren von Information, die sich auf Getränke beziehen
\end{itemize}
Diese fortschrittlichen Funktionen können ebenfalls im Unterkapitel \ref{preprocessing} nachgelesen werden.
Es wurden die Auswirkungen von jeglichen Kombinationen der fortgeschrittenen Preprocessingschritten getestet.
Bei vier verschiedenen Arten ergeben sich 16 verschiedene Kombinationen.
\subsubsection{Anzahl extrahierter Features}
Jeder Algorithmus verhaltet sich mit der Änderung der Anzahl Features unterschiedlich.
Um eine optimale Anzahl an Features zu ermitteln, musste schrittweise die Anzahl Features erhöht werden, die Algorithmen mit der neuen Anzahl trainiert und schlussendlich validiert werden.
Dies ist ein recht rechenintensives Unterfangen, führt in der Regel jedoch zu spürbaren Verbesserungen der Scores bei allen Algorithmen, sofern ein lokales Maximum der jeweiligen Algorithmen gefunden werden kann.
Die Messungen wurden mit einer Anzahl Features von 10 gestartet und schrittweise um 25 Features erhöht, bis schlussendlich eine maximale Anzahl von 400 Features erreicht wurde.
Da die Rechenzeit stetig mit der Erhöhung von Features steigt, wurde die Grenze bei 400 Features gesetzt.
Grundsätzlich macht eine Begrenzung der Features Sinn, da ab einer gewissen Anzahl Features die Algorithmen mit der Handhabung der Features überfordert sind und dies sich dementsprechend auf schlechteren Scores widerspiegelt.
\subsubsection{Erstellen der Konfigurationen}
\section{Produktive Pipeline}
\section{Webapplikation}
\subsection{Erarbeitung der Webapplikation}
\subsection{Verwendete Technologien}
\subsubsection{Frontend-Komponenten}
\subsubsection{Backend-Komponenten}
\subsection{Search Engine}
\subsubsection{Vorgehen}
\subsubsection{Verwendete Technologien}
\newpage

\include{Benutzung_Webapplikation}
\newpage


\chapter{Schlusswort}
\section{Erkenntnisse}
Während der Durchführung dieser Arbeit wurden wichtige Erkenntnisse gewonnen.
Eine Evaluation des Webcrawlers wäre von Vorteil gewesen.
StormCrawler hat zwar die Funktion erfüllt, dieser ist jedoch dafür ausgelegt, eine enorme Masse an Websites und Webpages zu crawlen.
Dies ist jedoch keine Hauptanforderung in dieser Arbeit gewesen.
Dafür wäre ein Webcrawler, welcher auch dynamisch generierte Websites handhaben kann, von Vorteil gewesen.
Die Implementierung der Spracherkennung innerhalb des Webcrawlers hat bei der ersten Implementierung für erhebliche Performanceprobleme gesorgt.
Diese wurde zwar optimiert, jedoch ist es prüfenswert, ob eine solche während des Crawlens überhaupt nötig ist.
Für die Klassifizierung hätte eine Preprocessing-Methode zur Erkennung des Hauptinhalts einer Webpage einen grossen Benefit bewirkt.
Beim manuellen Labeling wurde erkannt, dass viele Webpages nicht relevante oder sogar irreführende Informationen im Kopf- und Fussteil beinhalten.
%Erkenntnisse Machine Learning?
\section{Ausblick}
Als weiterer Schritt wäre eine Automatisierung wichtig.
Darunter versteht sich einerseits das Automatisieren des Webcrawlers, also eine kontinuierliche Erweiterung des Seeds sowie eine regelmässige Durchführung des Crawldurchlaufs.
Zudem würde die Performance erheblich gesteigert werden, wenn bei bereits gecrawlten Restaurants nur noch die als Menüseite klassifizierte Webpage gecrawlt werden würde und nicht die ganze Website.
Andererseits müssen die einzelnen Komponenten untereinander mittels definierter Schnittstellen verbunden und automatisiert werden, sodass gecrawlte Daten ohne manuellen Input klassifiziert und für die Webapplikation bereitgestellt werden.\\
Im Zusammenhang mit der Automatisierung muss auch die Performance optimiert werden.
Um beispielsweise tägliche Mittagsmenüs erfassen zu können, muss ein Crawldurchlauf innerhalb einiger Stunden durchgeführt werden.
Der Gold Standard sollte ein zweites Mal gelabelt werden, damit dieser nach dem Vier-Augen-Prinzip kontrolliert werden würde.
Um den F1-Score der Machine-Learning-Algorithmen zu erhöhen, wäre zudem eine Erweiterung des Gold Standards denkbar.


\newpage

\chapter{Selbstständigkeitserklärung}
\label{sec:erklaerung}
\vspace{0 cm}
Wir erklären hiermit, dass wir die vorliegende Arbeit selbstständig und ohne fremde Hilfe verfasst und keine anderen Hilfsmittel als angegeben verwendet haben. Insbesondere versichern wir, dass wir alle wörtlichen und sinngemässen Übernahmen aus anderen Werken als solche kenntlich gemacht haben.
\\
\\
Name: Sandro Santoro
\\	
Ort, Datum: \hspace{4cm}Unterschrift:
\\
\\
\\
Name: Gian Brunner
\\
Ort, Datum: \hspace{4cm}Unterschrift:
\\
\\





\newpage

%Literaturverzeichnis
\bibliographystyle{apacite}
\bibliography{Referenzen}

%Verzeichnis aller Tabellen
\listoftables

%Verzeichnis aller Bilder
\listoffigures

\chapter{Anhang}
https://github.com/s-santoro/lunch-crawler

\end{document}
