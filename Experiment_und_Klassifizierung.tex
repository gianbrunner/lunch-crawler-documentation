\chapter{Experiment und Klassifizierung}
\label{cap:exp_class}
Sämtliche Klassifikation findet anhand der Informationen der Attribute \glqq text\grqq{} und \glqq title\grqq{} des Gold Standards statt.
Zudem werden alle Webpages der Kategorie \glqq Tagesmenü\grqq{} der Kategorie \glqq Menü\grqq{} hinzugefügt, da bei dieser Klassifikation nur zwischen \glqq Menü\grqq{} oder \glqq Kein Menü\grqq{} unterschieden wird.
\section{Regelbasiertes Klassifizieren}
Jedes Regelset wurde anhand des Goldstandards getestet und evaluiert.
Verschiedene Parameterwerte und Kombinationen wurden getestet, um möglichst hohe Werte der gemessenen Metriken zu erreichen.
Bei allen Regelsets sind alle Methoden des Preprocessings aktiv gewesen. 
\subsection{Regelset: Menü im Titel}
Für dieses Regelset sind keine Parameter verfügbar, daher ist nur eine Konfiguration durchgführt worden.
Diese hat folgende Metriken ergeben:\\
\begin{table}[H]
\caption{Score des Regelsets: Menü im Titel}
\centering
\begin{tabular}{|l|l|l|}
	\hline
	F1-Score & Precision & Recall\\
	\hline
	0.17 & 0.43 & 0.11  \\
	\hline
\end{tabular}
\end{table}
\subsection{Regelset: Preisdetektor}
Durch die Konfiguration kann ein Schwellwert für die Anzahl erkannter Preise angegeben werden, die vorhanden sein müssen, um eine Webpage als positiv zu klassifizieren.\\
\begin{table}[H]
\caption{Scores des Regelsets: Preisdetektor}
\centering
\begin{tabular}{|l|l|l|l|}
	\hline
	Schwellwert & F1-Score & Precision & Recall\\
	\hline
	1 & 0.45 & 0.36 & 0.60  \\
	2 & 0.46 & 0.45 & 0.47 \\
	3 & 0.41 & 0.49 & 0.36 \\
	\hline
\end{tabular}
\end{table}
Das beste Ergebnis hat ein Schwellwert von zwei erzielt, danach ist der F1-Score wieder schlechter geworden.
Daraus wurde geschlussfolgert, dass für dieses Regelset das Maximum bereits erreicht wurde.
\subsection{Regelset: Kombination aus Menü im Titel und Preisdetektor}
Bei dieser Konfiguration ebenfalls kann der Schwellwert für die Anzahl erkannter Preise angegeben werden.\\
\begin{table}[H]
	\caption{Scores des Regelsets: Kombination aus Menü im Titel und Preisdetektor}
	\centering
\begin{tabular}{|l|l|l|l|}
	\hline
	Schwellwert & F1-Score & Precision & Recall\\
	\hline
	1 & 0.45 & 0.35 & 0.65 \\
	2 & 0.47 & 0.43 & 0.52 \\
	3 & 0.43 & 0.45 & 0.41 \\
	\hline
\end{tabular}
\end{table}
Auch bei diesem Regelset hat die Konfiguration mit einem Schwellwert von zwei das beste Ergebnis erzielt.
\subsection{Regelset: Listing}
Beim Listing können zwei Schwellwerte angeben werden, einen für die Anzahl übereinstimmender Wörter aus der Whitelist und einen für die Blacklist.
In einem ersten Versuch wurden identische Schwellwerte gewählt und jeweils erhöht:\\
\begin{table}[H]
	\caption{Scores der ersten Iteration des Regelsets: Listing}
	\centering
\begin{tabular}{|l|l|l|l|l|}
	\hline
	Schwellwert Whitelist & Schwellwert Blacklist & F1-Score & Precision & Recall\\
	\hline
	1 & 1 & 0.17 & 0.32 & 0.11 \\
	5 & 5 & 0.39 & 0.62 & 0.29 \\
	10 & 10 & 0.42 & 0.77 & 0.29 \\
	20 & 20 & 0.33 & 0.85 & 0.21 \\
	30 & 30 & 0.27 & 0.91 & 0.16 \\
	\hline
\end{tabular}
\end{table}
Dieser Versuch hat gezeigt, dass ein maximaler F1-Score bei gleichen Werten zwischen 5 und 20 zu erreichen ist.\\
Da gleich gewählte Werte keine zufriedenstellende Ergebnisse erzielten, wurden in zweiten Versuch wurden unterschiedliche Verhältnisse getestet:\\
\begin{table}[H]
	\caption{Scores der zweiten Iteration des Regelsets: Listing}
	\centering
\begin{tabular}{|l|l|l|l|l|}
	\hline
	Schwellwert Whitelist & Schwellwert Blacklist & F1-Score & Precision & Recall\\
	\hline
	5 & 1 & 0.12 & 0.82 & 0.07 \\
	1 & 5 & 0.31 & 0.24 & 0.45 \\
	\hline
\end{tabular}
\end{table}
Aus diesem Versuch entstand die Schlussfolgerung, dass ein höherer Schwellwert der Blacklist als der Whitelist erforderlich ist, um einen möglichst hohen Score zu erreichen.
Diese Erkenntnis wurde in einem weiteren Versuch in mehreren Iterationen getestet:\\
\begin{table}[H]
	\caption{Scores der dritten Iteration des Regelsets: Listing}
	\centering
\begin{tabular}{|l|l|l|l|l|}
	\hline
	Schwellwert Whitelist & Schwellwert Blacklist & F1-Score & Precision & Recall\\
	\hline
	2 & 10 & 0.42 & 0.33 & 0.61 \\
	3 & 15 & 0.52 & 0.42 & 0.70 \\
	3 & 20 & 0.51 & 0.39 & 0.75 \\
	4 & 15 & 0.53 & 0.47 & 0.61 \\
	5 & 15 & 0.55 & 0.54 & 0.56 \\
	5 & 20 & 0.54 & 0.50 & 0.60 \\
	5 & 21 & 0.54 & 0.49 & 0.60 \\
	6 & 20 & 0.55 & 0.56 & 0.55 \\
	7 & 20 & 0.55 & 0.60 & 0.50 \\
	8 & 20 & 0.53 & 0.64 & 0.46 \\
	\hline
\end{tabular}
\end{table}
Die Schwellwerte wurden stetig erhöht.
Verschiedene Schwellwerte in unterschiedlichen Verhältnissen wurden dabei getestet.
Sobald einer dieser Schwellwerte zu einem schlechteren Ergebnis geführt hat, wurde er wieder reduziert oder ein neuer Verhältnis wurde getestet.
Beim Verhältnis 6/20 bzw. 7/20 wurde das Maximum des F1-Scores erreicht.
Da diese Werte nicht im Bereich einer zufriedenstellenden Klassifikation sind, wurde auf das Ermitteln aller möglichen Kombinationen verzichtet.
\subsection{Regelset: Bag of Words}
Bei diesem Regelset kann die Grösse der Black- und Whitelist (Features), das Verhältnis zwischen Test- und Trainingsdaten (Split) sowie ein Schwellwert angegeben werden.
Für das Verhältnis zwischen Test- und Trainingsdaten wurden die Werte 0.3, 0.5 und 0.7 getestet.
In einer ersten Iteration wurde die Anzahl von 200 Features und ein Split von 0.3 verwendet, um herauszufinden, ob ein positiver oder negativer Schwellwert bessere Werte erzielt.\\
\begin{table}[H]
	\caption{Scores der ersten Iteration des Regelsets: Bag of Words}
	\centering
\begin{tabular}{|l|l|l|l|}
	\hline
	Schwellwert & F1-Score & Precision & Recall\\
	\hline
	0 & 0.64 & 0.58 & 0.71 \\
	2 & 0.51 & 0.38 & 0.75 \\
	-2 & 0.68 & 0.74 & 0.63 \\
	\hline
\end{tabular}
\end{table}
Dabei wurde erkannt, dass sich ein negativer Schwellwert positiv auf den F1-Score auswirkt.\\
In der zweiten Iteration wurde der Schwellwert weiter verkleinert.\\
\begin{table}[H]
	\caption{Scores der zweiten Iteration des Regelsets: Bag of Words}
	\centering
\begin{tabular}{|l|l|l|l|}
	\hline
	Schwellwert & F1-Score & Precision & Recall\\
	\hline
	-3 & 0.66 & 0.80 & 0.57 \\
	-4 & 0.66 & 0.86 & 0.54 \\
	-5 & 0.66 & 0.89 & 0.52 \\
	\hline
\end{tabular}
\end{table}\
Da diese Werte sich fast nicht unterscheiden, wurden alle weiterverwendet, um einen maximalen Score zu evaluieren.
In einer dritten, ausführlicheren Iteration sind die Anzahl Features von 200 - 400 sowie die drei oben genannten Verhältnisse zusammen mit den vier Schwellwerten getestet worden. Die Tabelle \cref{tab:bow3} zeigt diese Tests, sortiert nach bestem F1-Score:\\
\FloatBarrier
\begin{table}
	\caption{Scores der dritten Iteration des Regelsets: Bag of Words}
	\centering
	\label{tab:bow3}
\begin{tabular}{ | l | l | l | l | l | l | }
	\hline
	Split & Features & Limit & F1-Score & Precision & Recall \\ \hline
	0.3 & 400 & -3 & 0.72 & 0.74 & 0.7 \\ 
	0.3 & 400 & -4 & 0.72 & 0.79 & 0.66 \\
	0.3 & 400 & -5 & 0.72 & 0.81 & 0.64 \\
	0.7 & 400 & -3 & 0.72 & 0.75 & 0.69 \\
	0.7 & 400 & -4 & 0.72 & 0.79 & 0.66 \\
	0.5 & 300 & -3 & 0.71 & 0.79 & 0.65 \\
	0.5 & 400 & -3 & 0.70 & 0.74 & 0.67 \\
	0.5 & 400 & -4 & 0.70 & 0.79 & 0.63 \\ 
	0.7 & 400 & -2 & 0.70 & 0.70 & 0.70 \\ 
	0.7 & 300 & -3 & 0.70 & 0.74 & 0.67 \\
	0.7 & 300 & -4 & 0.70 & 0.80 & 0.63 \\
	0.7 & 300 & -5 & 0.70 & 0.84 & 0.60 \\
	0.7 & 400 & -5 & 0.70 & 0.82 & 0.62 \\ 
	0.3 & 400 & -2 & 0.69 & 0.66 & 0.73 \\ 
	0.5 & 400 & -2 & 0.69 & 0.69 & 0.69 \\ 
	0.5 & 400 & -5 & 0.69 & 0.82 & 0.60 \\ 
	0.7 & 200 & -3 & 0.69 & 0.76 & 0.63 \\ 
	0.7 & 200 & -4 & 0.69 & 0.81 & 0.60 \\ 
	0.3 & 300 & -2 & 0.68 & 0.73 & 0.64 \\ 
	0.3 & 300 & -3 & 0.68 & 0.78 & 0.60 \\ 
	0.3 & 200 & -2 & 0.68 & 0.74 & 0.63 \\ 
	0.5 & 200 & -3 & 0.68 & 0.77 & 0.60 \\ 
	0.5 & 300 & -4 & 0.68 & 0.81 & 0.59 \\ 
	0.7 & 300 & -2 & 0.68 & 0.69 & 0.68 \\ 
	0.5 & 200 & -2 & 0.67 & 0.71 & 0.64 \\ 
	0.3 & 300 & -4 & 0.66 & 0.82 & 0.56 \\ 
	0.3 & 200 & -3 & 0.66 & 0.80 & 0.57 \\
	0.3 & 200 & -4 & 0.66 & 0.86 & 0.54 \\
	0.3 & 200 & -5 & 0.66 & 0.89 & 0.52 \\ 
	0.5 & 200 & -4 & 0.66 & 0.81 & 0.56 \\
	0.5 & 300 & -5 & 0.66 & 0.83 & 0.55 \\ 
	0.7 & 200 & -2 & 0.66 & 0.66 & 0.66 \\ 
	0.7 & 200 & -5 & 0.66 & 0.85 & 0.55 \\ 
	0.3 & 300 & -5 & 0.65 & 0.85 & 0.52 \\ 
	0.5 & 200 & -5 & 0.65 & 0.87 & 0.52 \\
	0.5 & 300 & -2 & 0.46 & 0.62 & 0.36 \\ \hline
\end{tabular}
\end{table}
Daraus lässt sich schliessen, dass eine hohe Anzahl Features zu einem besseren Ergebnis führt.
Das Verhältnis zwischen Test- und Trainingsdaten ist nicht so relevant, da sowohl das Verhältnis 0.3 als auch 0.7 zu hohen Scores führt.
Der Schwellwert ist im Bereich -2 bis -5 ebenfalls nicht aussagekräftig, da auch dieser bei den besten Scores vertreten ist.
Es muss zudem berücksichtigt werden, dass der Split zufällig gewählt wird und keine Kreuzvalidierung stattfindet, dadurch können diese Ergebnisse variieren.
\FloatBarrier
\section{Auswirkungen des Preprocessings}
\subsection{Regelbasiertes Klassifizieren}
Bei den Methoden des regelbasierten Klassifizierens trägt das Preprocessing einen erheblichen Teil zum Erfolg bei.
Die Methode \glqq Menü im Titel\grqq{} profitiert davon, dass Umlaute mit den entsprechenden Selbstlauten ersetzt werden.
Der Preisdetektor funktioniert ohne den gleichnamigen Preprocessing-Schritt gar nicht, da nach dem Ersatzwort gesucht wird.
Beide Punkte gelten auch für die Kombination dieser Methoden.\\
Das Listing profitiert von mehreren Preprocessing-Schritten.
Das Ersetzen der Grossbuchstaben durch Kleinbuchstaben, der Preis- und Getränkedetektor sowie die Stammformreduktion führen dazu, dass im Text vorkommende Worte den Worten der jeweiligen Listen besser zugeordnet werden können.
Die Methode \glqq Bag of Words\grqq{} profitiert davon ebenfalls, da das Prinzip dasselbe ist.\\
Es können keine genauen Zahlen angegeben werden, welche Scores diese Methoden ohne Preprocessing-Schritte erreichen würden, da diese Schritte zwingend benötigt werden, um den Text in eine klassifizierbare Form zu bringen.
\subsection{Klassifizieren mittels Machine-Learning}
\subsubsection{Einfache Preprocessingschritte}
Das
\subsubsection{Fortschrittliche Preprocessingschritte}
\section{Klassifizieren mittels Machine-Learning}
\subsection{Dimensionsreduktion der Features}
\subsection{Klassenverteilung}
\subsection{N-Gramme}
\subsection{Anzahl extrahierter Features}
\subsection{Hyperparametertuning}
Die sechs Modelle, welche aus dem vorherigen Experiment, als die besten entnommen wurden, werden mittels Kreuzvalidierung auf optimale Hyperparameter durchsucht.
\subsubsection{Modelle mit bestem F1-Score}
Die drei Modelle AdaBoost, SGDClassifier und Perceptron konnten die besten F1-Scores erzielen.\\
\begin{table}
	\caption{Auwertung Hyperparametertuning für AdaBoost mit binärem Bag of Words}
	\centering
	\begin{tabular}{|l|l|l|l|}
		\hline
		 & F1-Score & Precision & Recall\\
		\hline
		Vor Hyperparametertuning & 0.778 & 0.837 & 0.727 \\
		Nach Hyperparametertuning & 0.788 & 0.817 & 0.761 \\
		\hline
	\end{tabular}
\end{table}
\begin{table}
	\caption{Auwertung Hyperparametertuning für Perceptron mit Bag of Words}
	\centering
	\begin{tabular}{|l|l|l|l|}
		\hline
		& F1-Score & Precision & Recall\\
		\hline
		Vor Hyperparametertuning & 0.8 & 0.872 & 0.739 \\
		Nach Hyperparametertuning & 0.579 & 0.426 & 0.903 \\
		\hline
	\end{tabular}
\end{table}
\begin{table}
	\caption{Auwertung Hyperparametertuning für SGDClassifier mit TF-IDF}
	\centering
	\begin{tabular}{|l|l|l|l|}
		\hline
		& F1-Score & Precision & Recall\\
		\hline
		Vor Hyperparametertuning & 0.805 & 0.768 & 0.847 \\
		Nach Hyperparametertuning & 0.762 & 0.682 & 0.864 \\
		\hline
	\end{tabular}
\end{table}
Auffällig ist, dass AdaBoost als einziges Modell bessere Hyperparameter mittels Hyperparemtertuning finden konnte.
AdaBoost konnte seinen Recall verbessern und gleichzeitig seine Precision verschlechtern.
Da die Recallsteigerung grösser als der Precisionabfall ist, wurde der F1-Score nach oben korrigiert.\\
Die anderen beiden Modelle erzielten mit den neuen Parametern schlechtere Werte.
Beide konnten den Recall verbessern, jedoch mussten sie massive Gefälle bei der Precision einbüssen.
Da die Precision viel stärker abgenommen, als der Recall zugenommen hat, wurde der F1-Score schlechter.\\
Perceptron hat im Vergleich zum SGDClassifier einen leicht tieferen F1-Score, jedoch ist seine Precision deutlich höher.
Da für den schlussendlichen \glqq Use-Case\grqq{} Precision wichtig ist, wird das Perceptron-Modell für weitere Auswertungen verwendet.
\subsubsection{Modelle mit bester Precision}
Das Modell RandomForest konnte für alle drei Feature-Extraction Methoden jeweils den besten Precision-Score erzielen.\\
\begin{table}
	\caption{Auwertung Hyperparametertuning für RandomForest mit binärem Bag of Words}
	\centering
	\begin{tabular}{|l|l|l|l|}
		\hline
		& F1-Score & Precision & Recall\\
		\hline
		Vor Hyperparametertuning & 0.636 & 1.0 & 0.466 \\
		Nach Hyperparametertuning & 0.736 & 0.8 & 0.682 \\
		\hline
	\end{tabular}
\end{table}
\begin{table}
	\caption{Auwertung Hyperparametertuning für RandomForest mit Bag of Words}
	\centering
	\begin{tabular}{|l|l|l|l|}
		\hline
		& F1-Score & Precision & Recall\\
		\hline
		Vor Hyperparametertuning & 0.636 & 1.0 & 0.466 \\
		Nach Hyperparametertuning & 0.768 & 0.829 & 0.716 \\
		\hline
	\end{tabular}
\end{table}
\begin{table}
	\caption{Auwertung Hyperparametertuning für RandomForest mit TF-IDF}
	\centering
	\begin{tabular}{|l|l|l|l|}
		\hline
		& F1-Score & Precision & Recall\\
		\hline
		Vor Hyperparametertuning & 0.747 & 0.956 & 0.614 \\
		Nach Hyperparametertuning & 0.794 & 0.823 & 0.767 \\
		\hline
	\end{tabular}
\end{table}
Alle drei Varianten konnten ihren F1-Score verbessern.
Die Verbesserung erfolgte nur im Bereich Recall, welcher initial bei allen drei Modellen relativ tief war.
Zusätzlich sanken bei allen drei Modellen die Precision-Scores.
Bei beiden Varianten mit Bag of Words, waren alle drei initialen Scores identisch und die Precision maximal.\\
Da in diesem Abschnitt auf die beste Precision geachtet wird, ist das Modell mit binärem Bag of Words und mit den Standardparametern die beste Variante.