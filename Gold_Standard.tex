\chapter{Teil 1: Gold Standard}
\label{chap:gold_standard}
Als Rohdaten zur Erhebung des Gold Standards dient der Output des Webcrawlers.
Die Erarbeitung dieser Rohdaten wird in \cref{chap:engineering} geschildert.
Der Link zum Gold Standard ist im Anhang (\cref{app:gold_standard}) zu finden.
\section{Seed}
Das Seed wurde als Datenquelle des Webcrawlers verwendet und bildet somit die Grundlage dieses Gold Standards.
Es wurde aus den folgenden zwei Quellen zusammengestellt:
\begin{itemize}
	\item OpenStreetMap - 3557 URLs zu Websites
	\item Lunch-Check - 3803 URLs zu Websites
\end{itemize}
Die entsprechenden Dateien sind im Anhang (\cref{app:electronic}) zu finden.\\
Diese Quellen wurden verwendet, da sie die Daten für diese Arbeit kostenlos zur Verfügung stellen.
Die URLs wurden zusammengeführt und Duplikate entfernt.
Daraus ist ein Seed entstanden, welches 5870 Einträge von Restaurant-URLs enthält.
Dabei wurden aus den nun aufgeführten Gründen mehrere Einträge entfernt:
\begin{itemize}
	\item Die Website enthält mehr als 300 Webpages
	\item Die Website ist offensichtlich keine Restaurant-Website
\end{itemize}
Websites mit mehr als 300 Webpages wurden entfernt, da diese keine typische Restaurant-Website repräsentieren und somit das Gesamtbild verzerren.
Es kann keine Gewähr gegeben werden, dass dieses Seed nur Restaurant-Websites beinhaltet, da nicht jeder Eintrag geprüft wurde.
\section{Entscheidungsraster}
Das Entscheidungsraster ist die Grundlage des manuellen Labeling der Rohdaten.
Für dieses wurden die folgenden Entscheidungen getroffen:
\begin{itemize}
	\item Die Webpage muss auf Deutsch verfasst sein
	\item Der Anbieter muss entweder ein Restaurant, Take-Away oder Lieferdienst sein
	\item Der Text muss statisch im HTML vorhanden sein
	\item Es muss ein Menü, also eine Kombination aus mehreren Speisen oder eine einzelne Speise, vorhanden sein
	\item Eine genauere Beschreibung oder der Preis muss vorhanden sein
	\item Getränkekarten werden explizit als negativ gelabelt
\end{itemize}
Bei der Klassifizierung wird zudem unterschieden, ob es sich um ein zeitlich begrenztes Angebot handelt, da diese Angebote zu einem späteren Zeitpunkt eventuell zusätzlich erkannt werden möchten.
Bei der Sprache ist anzumerken, dass gewisse Begriffe, vor allem für Speisebezeichnungen, auch fremdsprachig sein dürfen, da Speisebezeichnungen je nach Küche international ausgelegt sind.
Der Inhalt muss statisch im HTML verfügbar sein, da der Webcrawler nicht mit dynamisch gerenderten Websites umgehen kann.
Der Gold Standard wurde anhand des Entscheidungsrasters erstellt, welches in der \cref{fig:classificationtree} dargestellt wird.
\FloatBarrier
\begin{figure}
	\centering
	\includegraphics[width=0.75\columnwidth,keepaspectratio]{img/man-classification-tree.png}
	\caption{Entscheidungsraster}
	\source{Eigene Darstellung}
	\label{fig:classificationtree}
\end{figure}
\section{Manuelles Labeling der Daten}
In einer ersten Durchführung des manuellen Labelings wurden ca. 1500 Dateien klassifiziert, welche das Schlüsselwort \glqq Menu\grqq{} in der URL enthalten.
Die mit dieser Heuristik gefilterten Daten entsprechen jedoch nicht einer repräsentativen Teilmenge der gesamten Rohdaten.
Darum wurde in einer zweiten Durchführung zufällig Proben aus den Rohdaten ausgewählt und manuell gelabelt.
Dadurch konnte sichergestellt werden, dass das Verhältnis von Menüseiten zu den restlichen Webpages gleich bleibt.

\subsection{Hilfsmittel: Labeling-Tool}
Um das manuelle Labeling effizienter zu gestalten, ist ein Tool erstellt worden, welches das Labeling vereinfacht.
Dieses Tool ruft eine Webpage der zufällig extrahierten Webpages aus den Rohdaten auf und zeigt sowohl den Text, als auch den HTML-Inhalt.
Der Anwender des Tools muss anhand dieser Informationen entscheiden, ob es sich um eine Webpage mit Menüinformationen handelt und ob diese zeitlich begrenzt sind.
Mittels Shortkeys findet die Klassifizierung statt.
Das Tool verschiebt die Datei in den entsprechenden Ordner und zeigt die Informationen der nächsten Webpage an.
Der Link zum Tool ist im Anhang (\cref{app:gold_standard}) zu finden.
\FloatBarrier
\section{Beschreibung des Gold Standards}
Der erarbeitete Gold Standard besteht insgesamt aus 6963 Dateien des Formats JSON, welche jeweils eine Webpage einer Restaurant-Website repräsentieren.
Diese Daten sind in einen Test- und einen Trainings- und Validierungsdatensatz unterteilt.
Der Gold Standard beinhaltet die folgenden drei Kategorien:
\begin{itemize}
	\item neg $\rightarrow$ Keine Menüseite
	\item pos\textunderscore daily\textunderscore menu $\rightarrow$ Zeitlich ändernde Menüseite
	\item pos\textunderscore menu $\rightarrow$ Menüseite
\end{itemize}
Die Datenstruktur des Gold Standards sieht wie folgt aus:\\

\begin{forest}
	for tree={
		font=\ttfamily,
		grow'=0,
		child anchor=west,
		parent anchor=south,
		anchor=west,
		calign=first,
		edge path={
			\noexpand\path [draw, \forestoption{edge}]
			(!u.south west) +(7.5pt,0) |- node[fill,inner sep=1.25pt] {} (.child anchor)\forestoption{edge label};
		},
		before typesetting nodes={
			if n=1
			{insert before={[,phantom]}}
			{}
		},
		fit=band,
		before computing xy={l=15pt},
	}
	[Gold Standard (6963 Dateien)
	[test (100 Dateien)
	[neg (50 Dateien)]
	[pos\textunderscore daily\textunderscore menu (10 Dateien)]
	[pos\textunderscore menu (40 Dateien)]
	]
	[train-validation (6863 Dateien)
	[neg (6257 Dateien)]
	[pos\textunderscore daily\textunderscore menu (87 Dateien)]
	[pos\textunderscore menu (519 Dateien)]
	]
	]
\end{forest}\\


Jeder Eintrag dieses Gold Standards beinhaltet die folgenden Informationen:
\begin{itemize}
	\item \glqq date\grqq{} - Zeitpunkt, zu welchem die Webpage aufgerufen wurde
	\item \glqq text\grqq{} - Vom Webcrawler extrahierter Text, welcher die Webpage beinhaltet
	\item \glqq encoding\grqq{} - Das von der Webpage verwendete Encoding
	\item \glqq title\grqq{} - Inhalt des gleichnamigen HTML-Metatags
	\item \glqq url\grqq{} - URL der Webpage
	\item \glqq content\grqq{} - Der statische HTML-Inhalt der Webpage	
\end{itemize}


