\chapter{Einleitung}
Viele Menschen nutzen das Internet, um sich über Restaurants, deren Speisekarten und Mittagsmenüs zu informieren.
Diese Arbeit hat das Ziel, den Prototypen einer Suchmaschine zu entwickeln, über welche Restaurant-übergreifend Speisen und Menüs gesucht werden können.\\
Dazu werden verschiedene Teilkomponenten entwickelt, welche zusammen dieses Ziel erfüllen.
Bei der ersten Komponente handelt es sich um einen teilautomatisierten Webcrawler, welcher mit dem Software Development Kit \glqq StromCrawler\grqq{} erstellt wurde.
Dieser ruft Websites aus einem Seed, bestehend aus URLs von Restraunt-Websites, auf, findet deren Webpages  und speichert sie, sofern die Website dies erlaubt.
Durch diesen Webcrawler ist ein Rohdatensatz entstanden, der aus ca. 70'000 Webpages besteht.\\
Diese Rohdaten werden mittels einer weiteren Komponente, dem Classifier, klassifiziert, ob es sich dabei um Menüseiten bzw. Speisekarten handelt.
Diese Aufgabe wird mit zwei verschiedenen Ansätzen in einem Experiment versucht zu lösen.
Der erste Ansatz basiert auf regelbasierter Klassifikation.
Dabei wurden verschiedene Regeln erstellt, welche die Webpages Anhand des Textes und der Titels einer Webpage klassifizieren.
In einem zweiten Ansatz wird diese Aufgabe mit Methoden des maschinellen Lernens (Machine Learning) sowie der Verarbeitung natürlicher Sprache (NLP) versucht zu lösen.
Um die Algorithmen des maschinellen Lernens trainieren und die Ergebnisse der Klassifikation messen zu können, ist ein Gold Standard mit knapp 7000 von Hand kategorisierten Webpages erstellt worden.\\
Die Rohdaten werden mit dem Algorithmus, der die besten Ergebnisse liefert, klassifiziert.
Diejenigen Webpages, welche als Menüseite oder Speisekarte klassifiziert wurden, werden in einer Search Engine gespeichert und verwaltet.
Diese Komponente wird mit der Software ElasticSearch umgesetzt.
Die darin verwalteten Daten müssen zuvor auf einen einheitlichen Stand gebracht werden, damit jeder Eintrag Informationen über das Restaurant, die URLs der Menüseiten sowie der Geolocation enthält.\\
Als letzte Komponente kommt eine Webapplikation zum Einsatz. 
Mit dieser kann nach einer Speise gesucht werden und diese Suche optional mittels Umkreisfilter eingrenzen.
Dadurch lassen sich Restaurants, die eine spezifische Speise anbieten und in einem gewünschten Umkreis liegen, auf einer Karte anzeigen.
Die URL der Menüseite wird dabei verlinkt, sodass ein Benutzer diese direkt aufrufen kann.\\
Diese Arbeit beinhaltet zwei verschiedene Teile, einen wissenschaftlichen und einen praktischen Teil.
Im wissenschaftlichen Teil wird ein Gold Standard sowie die Klassifikation von dessen methodisch erarbeitet.
Im praktischen Teil werden die klassifizierten Webpages über eine Webapplikation verfügbar gemacht.