\chapter{Einleitung}
Viele Menschen nutzen das Internet, um sich über Restaurants, deren Speisekarten und Mittagsmenüs zu informieren.
Jedoch gibt es zur Zeit keine Suchmaschine, über welche sich Restaurant-übergreifend Speisen sowie Menüs suchen und finden lässt.
Diese Arbeit hat das Ziel, den Prototypen einer solchen Suchmaschine zu entwickeln.\\
Dazu werden verschiedene Teilkomponenten entwickelt, welche zusammen dieses Ziel erfüllen.
Bei der ersten Komponente handelt es sich um einen Webcrawler, welcher Websites automatisiert aufruft und speichert.
Jede Webpage dieser Websites wird dann mittels einer weiteren Komponente, dem Classifier, klassifiziert, ob es sich dabei um eine Menüseite oder Speisekarte handelt.
Webpages, welche als Menüseite oder Speisekarte klassifiziert wurden, werden in einer Search Engine gespeichert und verwaltet.
Als letzte Komponente kommt eine Webapplikation zum Einsatz. Ein Benutzer kann auf dieser Webapplikation nach einer Speise suchen und diese Suche optional mittels Standortfilter eingrenzen.\\
Diese Arbeit beinhaltet zwei verschiedene Teile, einen wissenschaftlichen und einen praktischen Teil.
Im wissenschaftlichen Teil wird ein Gold Standard sowie die Klassifikation von dessen methodisch erarbeitet.
Im praktischen Teil werden die klassifizierten Webpages über eine Webapplikation verfügbar gemacht.