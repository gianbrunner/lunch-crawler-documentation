\chapter{Forschungsfrage und Methodik}
\section{Forschungsfrage}
Aus dem Stand der Forschung ist die folgende Forschungsfrage entstanden:\\

\emph{Können Restaurant-Webseiten mit hoher Wahrscheinlichkeit klassifiziert werden, ob sie Menüinformationen beinhalten?}\\

Als hohe Erfolgschance wird aus Sicht der Autoren ein F1-Score von mindestens 0.8 definiert.
Es handelt sich um eine binäre Klassifikation, also eine Einteilung in zwei Kategorien, nämlich \glqq Menüseite\grqq{} oder \glqq Keine Menüseite\grqq.
Zudem sind die unter \cref{chap:scope} definierten Einschränkungen zu berücksichtigen.
\section{Methodik}
Um die Forschungsfrage beantworten zu können, werden zwei verschiedene Ansätze, namentlich das regelbasierte Klassifizieren sowie die Klassifikation mittels Machine-Learning angewendet.
Dies findet in mehreren Experimenten statt.
Um die Ergebnisse dieser Experimente messen und miteinander vergleichen zu können, wird ein Gold Standard erstellt.
\subsection{Gold Standard}
Als Startpunkt zur Erstellung des Gold Standards dient der Output des in \cref{chap:engineering} erstellten Webcrawlers.
Diese Daten werden von Hand kategorisiert, sodass ein repräsentativer Datensatz zur Messung der Klassifikation entsteht.
Um diese Daten kategorisieren zu können, müssen die Kategorien definiert und ein Entscheidungsraster erstellt werden.
Die zu klassifizierenden Daten werden zufällig ausgewählt, sodass eine natürliche Verteilung der jeweiligen Kategorien entsteht.
Die Daten werden zum Schluss in einen Trainings- und Testdatensatz unterteilt.
\subsection{Klassifikation}
Um die Daten des Gold Standards zu klassifizieren, werden sie zuerst mittels Preprocessing in eine standardisierte Form gebracht.
Die Klassifikation selbst wird in zwei verschiedene Kategorien unterteilt, das regelbasierte Klassifizieren sowie die Klassifikation mittels Machine-Learning.
Beim regelbasierten Klassifizieren werden fünf Methoden entwickelt, anhand welcher die Daten klassifiziert werden.
Bei der Klassifikation mittels Machine Learning werden die Algorithmen verwendet, welche im \cref{sec:algos} erläutert wurden.
Wegen des \glqq No-Free-Lunch\grqq{} Theorems werden die Algorithmen verschiedene Experimente durchlaufen und stetig miteinander verglichen.
Ziel der Experimente ist es, den besten Algorithmus mit der besten Konfiguration ausfindig zu machen.
Um einen möglichst hohen F1-Score zu erreichen, wird in beiden Fällen mit verschiedenen Konfigurationen gearbeitet.\\
Die Experimente werden bei jeglichen Klassifikationsmethoden anhand des Trainingsdatensatzes durchgeführt.
Um die schlussendlich gültigen Resultate zu erheben, wird der bis dahin unberührte Testdatensatz verwendet.
Die Ergebnisse werden anhand des F1-Scores gemessen.
Wenn zwei verschiedene Konfigurationen minimal unterschiedliche F1-Scores erreichen, wird diejenige Konfiguration bevorzugt, welche die höhere Precision erreicht.
