\chapter{Teil 2: Klassifikation}
\section{Einleitung}
% Pipeline erklären
\section{Preprocessing}
\subsection{Basis}
\subsection{Fortgeschritten}
\section{Experimente}
\subsection{Regelbasierte Experimente}
\subsubsection{Regelsatz: Menü im Titel}
\paragraph{Beschreibung der Komponente}
\paragraph{Methoden}
\paragraph{Resultate}
\subsubsection{Regelsatz: Preisdetektor}
\paragraph{Beschreibung der Komponente}
\paragraph{Methoden}
\paragraph{Resultate}
\subsubsection{Regelsatz: Kombination aus Menü im Titel und Preisdetektor}
\paragraph{Beschreibung der Komponente}
\paragraph{Methoden}
\paragraph{Resultate}
\subsubsection{Regelsatz: Listing}
\paragraph{Beschreibung der Komponente}
\paragraph{Methoden}
\paragraph{Resultate}
\subsubsection{Regelsatz: Bag of Words}
\paragraph{Beschreibung der Komponente}
\paragraph{Methoden}
\paragraph{Resultate}
\subsubsection{Diskussion der regelbasierten Experimente}
\subsection{Experimente mittels Machine-Learning}
\subsubsection{Dimensionsreduktion der Features}
\paragraph{Beschreibung der Komponente}
\paragraph{Methoden}
\paragraph{Resultate}
\subsubsection{Angabe von Klassenverteilung}
\paragraph{Beschreibung der Komponente}
\paragraph{Methoden}
\paragraph{Resultate}
\subsubsection{Anwendung von N-Gramme}
\paragraph{Beschreibung der Komponente}
\paragraph{Methoden}
\paragraph{Resultate}
\subsubsection{Anwendung von einfachen Preprocssingschritten}
\paragraph{Beschreibung der Komponente}
\paragraph{Methoden}
\paragraph{Resultate}
\subsubsection{Anwendung von fortgeschrittenen Preprocssingschritten}
\paragraph{Beschreibung der Komponente}
\paragraph{Methoden}
\paragraph{Resultate}
\subsubsection{Anzahl extrahierter Features}
\paragraph{Beschreibung der Komponente}
\paragraph{Methoden}
\paragraph{Resultate}
\subsubsection{Hyperparametertuning}
\paragraph{Beschreibung der Komponente}
\paragraph{Methoden}
\paragraph{Resultate}
\subsubsection{Diskussion der regelbasierten Experimente}
\section{Diskussion}
\section{Beantwortung der Forschungsfrage}