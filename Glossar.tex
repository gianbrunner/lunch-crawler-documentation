\addchap{Glossar}
Crawlen - Die Tätigkeit eines Webcrawlers: Das Aufrufen und Herunterladen von Webseiten\\
OSM - Opensteetmap, eine freie Geodatenbank\\
SDK - Software Development Kit (SDK) Programmierwerkzeug und Bibliothek, welches hilft, eine Software zu entwickeln\\
Seed - Eine Liste bestehend aus Webseiten, welche gecrawlt werden sollen\\
Stormcrawler - SDK zur Entwicklung eines Webcrawlers\\
Webcrawler - Programm zum automatisierten Aufrufen und Speichern einer Webseite\\
Website - Komplette Internetseite (Startseite inkl. allen Subwebsites) z.B. www.menucrawler.ch\\
Webpage - Spezifischer Teil einer Webseite z.B. www.menucrawler.ch/ueber-uns\\
idempotent - Bei gleicher Eingabe erfolgt stets die gleiche Ausgabe, egal wie oft die Aufgabe ausgeführt wird\\
Open-Source - Software, deren Quelltext frei einsehbar und änderbar ist\\
Big Data - Arbeitsumfeld, indem mit sehr vielen Datenpunkten gearbeitet wird\\
Machine-Learning - Maschinelles Lernen, der Computer wird darauf trainiert, eine Aufgabe zu verrichten\\
Pipelining - Das verbinden von mehreren Aufgaben zu einer Gesamtaufgabe\\
Stream - Kontinuierlicher Fluss von Datensätzen\\
URL - Uniform Ressource Locator - Eindeutiger Link zu einer Webpage\\
Blacklisting - Index, der zur Benachteiligung der darin aufgeführten Einträgen führt\\
Whitelisting - Index, der zur Bevorzugung der darin aufgeführten Einträgen führt\\
Labeling - Kategorisieren von Daten\\
JSON - JavaScript Object Notation, ein kompaktes Dateiformat in lesbarer Textform, geeignet für den Datenaustausch zwischen Anwendungen\\
No free lunch - Ein Theorem, welches besagt, dass es kein universelles Verfahren für die Lösung eines ML-Problems gibt\\
N-Gramme - Wörterkette mit Anzahl N Wörtern\\
Features - Merkmale, mit denen ein ML-Algorithmus trainiert werden kann\\
Feature-Extraction - Das Extrahieren von relevanten Features\\
LSA (Latent Sentiment Analysis) - Verfahren zum Auffinden von Schlüsselbegriffen\\
Bigramm - Eine Wörterkette mit Länge zwei
Trigramm - Eine Wörterkette mit Länge drei 