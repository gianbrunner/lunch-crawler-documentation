\addchap{Glossar}
\begin{table}[H]
	\begin{tabular}{>{\em}p{4cm}p{12cm}}
		API & Application Programming Interface - Applikationsprogrammierschnittstelle\\
		Bigramm & Eine Wörterkette mit Länge Zwei\\
		Big Data & Arbeitsumfeld, indem mit sehr vielen Datenpunkten gearbeitet wird\\
		Blacklisting & Index, der zur Benachteiligung der darin aufgeführten Einträgen führt\\
		BSD & Berkeley Software Distribution - Eine Softwarelizenz, welche die Weiterbearbeitung von Quelltext erlaubt\\
		Crawlen & Die Tätigkeit eines Webcrawlers: Das Aufrufen und Herunterladen von Webseiten\\
		Data Mining & Behandlung von grossen Datenmengen um Zusammenhänge zu finden (z.B. Kaufverhalten von Nutzern anhand von Suchverläufen herausfinden)\\
		F1-Score & Metrik, welche der harmonische Durchschnitt von Precision und Recall ist\\
		Fast Prototyping & Das schnelle Entwickeln eines Prototypen, welche die Grundfunktionen mehr oder weniger befriedigend ausführt\\
		Features & Merkmale, mit denen ein ML-Algorithmus trainiert werden kann\\
		Feature-Extraction & Das Extrahieren von relevanten Features\\
		Geolocation & Geografische Koordinaten eines Objekts\\
		Geocoding & Ermitteln der Geolocation durch Adressinformationen\\
		HTML & Hypertext Markup Language - Auszeichnungssprache zur Strukturierung elektronischer Dokumente, Grundlage des World Wide Web\\
		Idempotenz & Bei gleicher Eingabe erfolgt stets die gleiche Ausgabe, auch bei mehrmaligem Durchführen\\
		JSON & JavaScript Object Notation - ein kompaktes Dateiformat in lesbarer Textform, geeignet für den Datenaustausch zwischen Anwendungen\\
		Labeling & Kategorisieren von Daten\\
		LSA & Latent Sentiment Analysis - Verfahren zum Auffinden von Schlüsselbegriffen\\
		Machine-Learning & Maschinelles Lernen, der Computer wird darauf trainiert, eine Aufgabe zu verrichten\\
		N-Gramme & Wörterkette mit Anzahl N Wörtern\\
		NLP & Natural Language Processing - Die Verarbeitung natürlicher Sprache\\
		No free lunch & Ein Theorem, welches besagt, dass es kein universelles Verfahren für die Lösung eines ML-Problems gibt\\
		
	\end{tabular}
\end{table}
\begin{table}[H]
	\begin{tabular}{>{\em}p{4cm}p{12cm}}
		Opensource & Software, deren Quelltext frei verfügbar und änderbar ist\\
		OSM & OpenStreetMap - Eine freie Geodatenbank\\
		Overfitting & Bezeichnung im Machine-Learning, wenn Modelle eine Überanpassung der Daten durchführen. Bei einer Überanpassung lernt das Modell die Trainingsdaten auswendig, kann aber sehr schlecht Generalisierungsentscheidungen fällen.\\
		Pipelining & Das verbinden von mehreren Aufgaben zu einer Gesamtaufgabe\\
		Precision & Metrik, die angibt, wie viele gefundene Artikel auch relevant sind\\
		Recall & Metrik, die angibt, wie viele relevante Artikel gefunden worden sind\\
		Scores & Resultate der Klassifizierer\\
		SDK & Software Development Kit - Programmierwerkzeug und Bibliothek, welches hilft, eine Software zu entwickeln\\
		Seed & Eine Liste bestehend aus Webseiten, welche gecrawlt werden sollen\\
		Stormcrawler & SDK zur Entwicklung eines Webcrawlers\\
		Stream & Kontinuierlicher Fluss von Datensätzen\\
		Supervised Learning & Das maschinelle Lernen mit Daten, welche im Voraus markiert worden sind (z.B. manuelles Markieren von Spam bei Emails)\\
		Trigramm & Eine Wörterkette mit Länge drei\\
		Unsupervised Learning & Das maschinelle Lernen mit Daten, welche nicht markiert sind. Dadurch kann der Computer keine konkrete Klassifizierung machen, sondern nur Ähnlichkeiten in den Daten aufzeigen\\
		URL & Uniform Ressource Locator - Eindeutiger Link zu einer Webpage\\
		Webcrawler & Programm zum automatisierten Aufrufen und Speichern einer Webseite\\
		Webpage & Spezifischer Teil einer Webseite z.B. www.menucrawler.ch/ueber-uns\\
		Website & Komplette Internetseite (Startseite inkl. aller Webpages) z.B. www.menucrawler.ch\\
		Whitelisting & Index, der zur Bevorzugung der darin aufgeführten Einträgen führt\\	
	\end{tabular}
\end{table}
