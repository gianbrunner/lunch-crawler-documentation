\addchap{Abstract}
\addsec{Abstract Deutsch}

Ziel dieser Arbeit ist das Erstellen einer Suchmaschine, über welche sich Menüs und Speisen suchen lassen.
Die Grundlage einer solchen Suchmaschine sind Websites von Restaurants, welche relevante Speiseinformationen beinhalten.
Im Kontext dieser Bachelorarbeit wurde manuell ein Gold-Standard aus Restaurantseiten zusammengestellt.
Für die Erstellung des Gold-Standards wurde eigens ein Webcrawler implementiert, welcher eine Vielzahl von Restaurant-Links besucht und den Webseiteninhalt abspeichert.
Der erstellte Gold-Standard dient dazu, eine Klassifikation der Restaurantseiten anhand zwei verschiedener Ansätze durchzuführen und zu messen.
Die zwei Ansätze sind regelbasiertes Klassifizieren sowie das Klassifizieren mittels Machine-Learning.
Um die einzelnen Klassifikationen prüfen zu können, wurden in beiden Bereichen mehrere Experimente durchgeführt.
Im praktischen Teil der Arbeit wurde neben dem Webcrawler eine Webapplikation erarbeitet, welche die Suchmaschine den Benutzern zugänglich macht.
\addsec{Abstract English}
The aim of this work is to create a search engine, which searches menus and meals across restaurants.
The basis of this search engine are websites of restaurants.
To create a data set of such websites, a web crawler has been implemented.
In the scientific part of this work, a gold standard was created by hand from this data set.
This serves to execute and measure a classification of the restaurant web pages using two different approaches, namely rule-based classification and machine learning based classification.
To test these approaches, several experiments were performed.
In the practical part, a web application was developed in addition to the web crawler, which serves as a search engine and a user presents the classified data.
