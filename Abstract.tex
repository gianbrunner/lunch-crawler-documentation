\addchap{Abstract}
\addsec{Abstract Deutsch}
Die Idee dieser Arbeit ist das Erstellen einer Suchmaschine, über welche sich Restaurant-übergreifend Menüs und Speisen suchen lassen.
Die Grundlage einer solchen Suchmaschine sind die Websites von Restaurants in Kombination mit Informationen wie z.B. dem Standort.
Um einen Datensatz solcher Websites zu erstellen, ist ein Webcrawler implementiert worden.
Im wissenschaftlichen Teil dieser Arbeit wurde anhand dieses Datensatzes ein Gold Standard erstellt.
Dieser dient dazu, die eigentliche Forschungsfrage beantworten zu können, die folgendermassen lautet:\\
\emph{Können Restaurant-Webseiten mit hoher Wahrscheinlichkeit klassifiziert werden, ob sie Menüinformationen beinhalten?}\\
Vorausgesetzt wird ein F1-Score von 0.8 oder höher, um diese Frage mit \glqq Ja\grqq{} beantworten zu können.
Dabei werden zwei verschiedene Ansätze zur Klassifikation angewandt, namentlich das regelbasierte Klassifizieren sowie das Klassifizieren mittels Machine-Learning.
Um diese Ansätze prüfen zu können, werden mehrere Experimente durchgeführt.
Zum Schluss wird erkannt, dass der gewünschte F1-Score mit beiden Ansätzen knapp verfehlt wurde.
Im praktischen Teil wird neben dem Webcrawler eine Webapplikation erarbeitet, welche als Suchmaschine dient und die klassifizierten Daten dem Benutzer darstellt.
\addsec{Abstract English}
The idea of this work is to create a search engine that can be used to search menus and meals restaurant independently.
The basis of such a search engine are the websites of restaurants in combination with information such as, for example, the location.
To create a dataset of such websites, a web crawler has been implemented.
In the scientific part of this work, a gold standard was created based on this dataset.
This serves to answer the actual research question, which is the following: \\
\emph{Can Webpages of restaurants be classified with high chance of success, if they contain menu information?} \\
An F1 score of 0.8 or higher is required to answer this question with \glqq Yes\grqq{}.
Two different approaches are used for classification, in particular rule-based classification and classification with machine learning.
To be able to test these approaches, several experiments are carried out.
Finally, it is recognized that the desired F1 score was missed with both approaches.
In the practical part, in addition to the webcrawler, a web application will be developed which serves as a search engine and displays the classified data to a user.