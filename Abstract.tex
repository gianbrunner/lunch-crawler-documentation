\addchap{Abstract}
\addsec{Abstract Deutsch}

Ziel dieser Arbeit ist das Erstellen einer Suchmaschine, über welche sich Menüs und Speisen suchen lassen.
Die Grundlage einer solchen Suchmaschine sind Websites von Restaurants, welche relevante Speiseinformationen beinhalten.
Im Kontext dieser Bachelorarbeit wurde manuell ein Gold-Standard aus Restaurantseiten zusammengestellt.
Für die Erstellung des Gold-Standards wurde eigens ein Webcrawler implementiert, welcher eine Vielzahl von Restaurant-Links besucht und den Webseiteninhalt abspeichert.
Der erstellte Gold-Standard dient dazu, eine Klassifikation der Restaurantseiten anhand zwei verschiedener Ansätze durchzuführen und zu messen.
Die zwei Ansätze sind regelbasiertes Klassifizieren sowie das Klassifizieren mittels Machine-Learning.
Um die einzelnen Klassifikationen prüfen zu können, wurden in beiden Bereichen mehrere Experimente durchgeführt.
Im praktischen Teil der Arbeit wurde neben dem Webcrawler eine Webapplikation erarbeitet, welche die Suchmaschine den Benutzern zugänglich macht.
\addsec{Abstract English}
The idea of this work is to create a search engine that can be used to search menus and meals restaurant independently.
The basis of such a search engine are the websites of restaurants in combination with information such as, for example, the location.
To create a dataset of such websites, a web crawler has been implemented.
In the scientific part of this work, a gold standard was created based on this dataset.
This serves to answer the actual research question, which is the following: \\
\emph{Can Webpages of restaurants be classified with high chance of success, if they contain menu information?} \\
An F1 score of 0.8 or higher is required to answer this question with \glqq Yes\grqq{}.
Two different approaches are used for classification, in particular rule-based classification and classification with machine learning.
To be able to test these approaches, several experiments are carried out.
Finally, it is recognized that the desired F1 score was missed with both approaches.
In the practical part, in addition to the webcrawler, a web application will be developed which serves as a search engine and displays the classified data to a user.