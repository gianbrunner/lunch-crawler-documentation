\addchap{Abstract}
\addsec{Abstract Deutsch}
Ziel dieser Arbeit ist das Erstellen einer Suchmaschine, über welche sich Menüs und Speisen suchen lassen.
Die Grundlage einer solchen Suchmaschine sind Websites von Restaurants.
Um einen Datensatz solcher Websites zu erstellen, ist ein Webcrawler implementiert worden.
Im wissenschaftlichen Teil dieser Arbeit wurde anhand dieses Datensatzes von Hand ein Gold Standard erstellt.
Dieser dient dazu, eine Klassifikation der Restaurant-Webpages anhand zwei verschiedener Ansätze durchzuführen und zu messen, namentlich dem regelbasierte Klassifizieren sowie dem Klassifizieren mittels Machine-Learning.
Um diese Ansätze prüfen zu können, wurden mehrere Experimente durchgeführt.
Im praktischen Teil wurde neben dem Webcrawler eine Webapplikation erarbeitet, welche als Suchmaschine dient und einem Benutzer die klassifizierten Daten darstellt.

\addsec{Abstract English}
The aim of this work is to create a search engine, which searches menus and meals across restaurants.
The basis of this search engine are websites of restaurants.
To create a data set of such websites, a web crawler has been implemented.
In the scientific part of this work, a gold standard was created by hand from this data set.
This serves to execute and measure a classification of the restaurant web pages using two different approaches, namely rule-based classification and machine learning based classification.
To test these approaches, several experiments were performed.
In the practical part, a web application was developed in addition to the web crawler, which serves as a search engine and a user presents the classified data.