\addchap{Abstract}
\addsec{Abstract Deutsch}
Ziel dieser Arbeit ist das Erstellen einer Suchmaschine, über welches sich Menüs und Speisen suchen lassen.
Die Grundlage einer solchen Suchmaschine sind Websites von Restaurants, welche relevante Speiseinformationen beinhalten.
Im Kontext dieser Bachelorarbeit wurde manuell ein Gold Standard aus Restaurantseiten zusammengestellt.
Für die Erstellung des Gold Standards wurde eigens ein Webcrawler implementiert, welcher eine Vielzahl von Restaurant-Links aufruft und den Webseiteninhalt abspeichert.
Der erstellte Gold Standard dient dazu, eine Klassifikation der Restaurantseiten anhand zwei verschiedener Ansätze durchzuführen und zu messen.
Die zwei Ansätze sind regelbasiertes Klassifizieren sowie das Klassifizieren mittels Machine-Learning.
Um die einzelnen Klassifikationen prüfen zu können, wurden in beiden Bereichen mehrere Experimente durchgeführt.
In den Experimenten wird aufgezeigt, dass die Klassifikation mittels Machine-Learning einen F1-Score von 0.78 erreicht und weiteres Potential aufweist, wogegen ein regelbasierter Ansatz nicht über einen F1-Score von 0.7 kommt.
Im praktischen Teil der Arbeit wurde zusätzlich zum Webcrawler eine Webapplikation erarbeitet, welche die Suchmaschine den Benutzern zugänglich macht.
\addsec{Abstract English}
The aim of this work is to create a search engine, which searches for menus and meals.
The basis of this search engine are websites of restaurants, which contain relevant food information.
In the context of this bachelor thesis, a gold standard has been created manually from restaurant webpages.
To get the data for this gold standard, a webcrawler was implemented, which visits a large number of restaurant links and saves the website content.
This gold standard is used to execute and measure a classification of the restaurant webpages using two different approaches.
The two approaches are rule-based classification and classifying by means of machine learning.
In order to measure both classification outcomes, several experiments were executed for both approaches.
In these experiments, it is shown that the classification with machine learning achieves an F1 score of 0.78 and has further potential, whereas a rule-based approach only reaches an F1 score of 0.7.
In the practical part of the work, a web application was developed in addition to the web crawler.
This web application makes the search engine accessible to users.
